%% Generated by Sphinx.
\def\sphinxdocclass{report}
\documentclass[a4paper,10pt,english]{sphinxmanual}
\ifdefined\pdfpxdimen
   \let\sphinxpxdimen\pdfpxdimen\else\newdimen\sphinxpxdimen
\fi \sphinxpxdimen=.75bp\relax
%% turn off hyperref patch of \index as sphinx.xdy xindy module takes care of
%% suitable \hyperpage mark-up, working around hyperref-xindy incompatibility
\PassOptionsToPackage{hyperindex=false}{hyperref}

\PassOptionsToPackage{warn}{textcomp}


\usepackage{cmap}
\usepackage{xeCJK}
\usepackage{amsmath,amssymb,amstext}
\usepackage[english]{babel}



\setmainfont{FreeSerif}[
  Extension      = .otf,
  UprightFont    = *,
  ItalicFont     = *Italic,
  BoldFont       = *Bold,
  BoldItalicFont = *BoldItalic
]
\setsansfont{FreeSans}[
  Extension      = .otf,
  UprightFont    = *,
  ItalicFont     = *Oblique,
  BoldFont       = *Bold,
  BoldItalicFont = *BoldOblique,
]
\setmonofont{FreeMono}[
  Extension      = .otf,
  UprightFont    = *,
  ItalicFont     = *Oblique,
  BoldFont       = *Bold,
  BoldItalicFont = *BoldOblique,
]


\usepackage[Sonny]{fncychap}
\ChNameVar{\Large\normalfont\sffamily}
\ChTitleVar{\Large\normalfont\sffamily}
\usepackage{sphinx}

\fvset{fontsize=\small}
\usepackage{geometry}

% Include hyperref last.
\usepackage{hyperref}
% Fix anchor placement for figures with captions.
\usepackage{hypcap}% it must be loaded after hyperref.
% Set up styles of URL: it should be placed after hyperref.
\urlstyle{same}

\usepackage{sphinxmessages}
\setcounter{tocdepth}{1}


        \usepackage{ctex}
        

\title{mydoc}
\date{Nov 11, 2019}
\release{0.1}
\author{Xiaolaing Dai}
\newcommand{\sphinxlogo}{\vbox{}}
\renewcommand{\releasename}{Release}
\makeindex
\begin{document}

\ifdefined\shorthandoff
  \ifnum\catcode`\=\string=\active\shorthandoff{=}\fi
  \ifnum\catcode`\"=\active\shorthandoff{"}\fi
\fi

\pagestyle{empty}
\sphinxmaketitle
\pagestyle{plain}
\sphinxtableofcontents
\pagestyle{normal}
\phantomsection\label{\detokenize{index::doc}}

\bigskip\hrule\bigskip


Contents:




\chapter{系统相关}
\label{\detokenize{linux/shell:id1}}\label{\detokenize{linux/shell::doc}}

\section{SSH 协议}
\label{\detokenize{linux/shell:ssh}}\begin{quote}

SSH协议是建立在不安全的网络之上的进行远程安全登陆的协议。它是一个协议族,其中有三个子协议,分别是:
\begin{itemize}
\item {} 
传输层协议 {[}SSH-TRANS{]}: 提供 服务器 验证、完整性和保密性功能,建立在传统的TCP/IP协议之上。

\item {} 
验证协议{[}SSH-USERAUTH{]}: 向服务器验证客户端用户,有基于用户名密码和公钥两种验证方式,建立在传输层协议 {[}SSH-TRANS{]} 之上。

\item {} 
连接协    {[}SSH-CONNECT: 将加密隧道复用为若干逻辑信道。它建立在验证协议之上。

\end{itemize}
\begin{quote}

\noindent\sphinxincludegraphics{{ssh1}.png}
\end{quote}

\sphinxstylestrong{握手过程:}
\begin{quote}

\noindent\sphinxincludegraphics{{ssh2}.png}
\end{quote}
\begin{itemize}
\item {} 
三次握手

\end{itemize}
\begin{quote}

\noindent\sphinxincludegraphics{{ssh_cap1}.png}
\end{quote}
\begin{itemize}
\item {} 
协议交换

\end{itemize}
\begin{quote}

\noindent\sphinxincludegraphics{{ssh_cap2}.png}
\end{quote}
\begin{itemize}
\item {} 
密钥交换

\end{itemize}
\begin{quote}

\noindent\sphinxincludegraphics{{ssh_cap3}.png}
\end{quote}
\end{quote}


\section{SSH 免密}
\label{\detokenize{linux/shell:id2}}\begin{quote}
\begin{itemize}
\item {} 
服务端配置
\begin{quote}

生成公钥和私钥:ssh-keygen -t rsa
\end{quote}

\item {} 
客户端端配置
\begin{quote}

\begin{DUlineblock}{0em}
\item[] 创建目录:/root/.ssh/
\item[] 将发送端公钥复制到此目录下,并重命名为authorized\_keys
\end{DUlineblock}
\end{quote}

\end{itemize}

\begin{sphinxVerbatim}[commandchars=\\\{\}]
\PYG{c+c1}{\PYGZsh{}使用ssh\PYGZus{}copy}
ssh\PYGZhy{}copy\PYGZhy{}id \PYGZhy{}i .ssh/id\PYGZus{}rsa.pub  用户名字@192.168.x.xxx
\end{sphinxVerbatim}
\begin{itemize}
\item {} 
SSH批量免密

\end{itemize}
\begin{quote}

\fvset{hllines={, 9,}}%
\begin{sphinxVerbatim}[commandchars=\\\{\},numbers=left,firstnumber=1,stepnumber=1]
\PYG{c+ch}{\PYGZsh{}!/bin/bash}
\PYG{n+nv}{SERVERS}\PYG{o}{=}\PYG{l+s+s2}{\PYGZdq{}192.168.1.241 192.168.1.242\PYGZdq{}}
\PYG{n+nv}{PASSWD}\PYG{o}{=}\PYG{l+s+s2}{\PYGZdq{}123456\PYGZdq{}}
 
\PYG{k}{function} sshcopyid
\PYG{o}{\PYGZob{}}
    expect \PYGZhy{}c \PYG{l+s+s2}{\PYGZdq{}}
\PYG{l+s+s2}{        set timeout \PYGZhy{}1;}
\PYG{l+s+s2}{        spawn ssh\PYGZhy{}copy\PYGZhy{}id }\PYG{n+nv}{\PYGZdl{}1}\PYG{l+s+s2}{;}
\PYG{l+s+s2}{        expect \PYGZob{}}
\PYG{l+s+s2}{            \PYGZbs{}\PYGZdq{}yes/no\PYGZbs{}\PYGZdq{} \PYGZob{} send \PYGZbs{}\PYGZdq{}yes\PYGZbs{}r\PYGZbs{}\PYGZdq{} ;exp\PYGZus{}contine; \PYGZcb{}}
\PYG{l+s+s2}{            \PYGZbs{}\PYGZdq{}password:\PYGZbs{}\PYGZdq{} \PYGZob{} send \PYGZbs{}\PYGZdq{}}\PYG{n+nv}{\PYGZdl{}PASSWD}\PYG{l+s+s2}{\PYGZbs{}r\PYGZbs{}\PYGZdq{};exp\PYGZus{}continue; \PYGZcb{}}
\PYG{l+s+s2}{        \PYGZcb{};}
\PYG{l+s+s2}{        expect eof;}
\PYG{l+s+s2}{    }\PYG{l+s+s2}{\PYGZdq{}}
\PYG{o}{\PYGZcb{}}
 
\PYG{k}{for} server in \PYG{n+nv}{\PYGZdl{}SERVERS}
\PYG{k}{do}
    sshcopyid \PYG{n+nv}{\PYGZdl{}server}
\PYG{k}{done}
\end{sphinxVerbatim}
\sphinxresetverbatimhllines
\end{quote}
\end{quote}


\section{磁盘IO测试}
\label{\detokenize{linux/shell:io}}\begin{quote}

\begin{sphinxVerbatim}[commandchars=\\\{\}]
\PYG{c+c1}{\PYGZsh{}使用ssh\PYGZus{}copy}
ssh\PYGZhy{}copy\PYGZhy{}id \PYGZhy{}i .ssh/id\PYGZus{}rsa.pub  用户名字@192.168.x.xxx
\PYG{c+c1}{\PYGZsh{}!/bin/bash}
\PYG{c+c1}{\PYGZsh{}随机写:}
fio \PYGZhy{}filename\PYGZhy{}/dev/sda1 \PYGZhy{}direct\PYGZhy{}1 \PYGZhy{}iodepth \PYG{l+m}{1} \PYGZhy{}thread \PYGZhy{}rw\PYGZhy{}randwrite \PYGZhy{}ioengine\PYGZhy{}psync \PYGZhy{}bs\PYGZhy{}4k \PYGZhy{}size\PYGZhy{}2G \PYGZhy{}numjobs\PYGZhy{}10 \PYGZhy{}runtime\PYGZhy{}60 \PYGZhy{}group\PYGZus{}reporting \PYGZhy{}name\PYGZhy{}mytest
\PYG{c+c1}{\PYGZsh{}随机读:}
fio \PYGZhy{}filename\PYGZhy{}/2G.fille  \PYGZhy{}direct\PYGZhy{}1 \PYGZhy{}iodepth \PYG{l+m}{1} \PYGZhy{}thread \PYGZhy{}rw\PYGZhy{}randread \PYGZhy{}ioengine\PYGZhy{}psync \PYGZhy{}bs\PYGZhy{}16k \PYGZhy{}size\PYGZhy{}2G \PYGZhy{}numjobs\PYGZhy{}10 \PYGZhy{}runtime\PYGZhy{}60 \PYGZhy{}group\PYGZus{}reporting \PYGZhy{}name\PYGZhy{}mytest
\PYG{c+c1}{\PYGZsh{}混合读写:}
fio \PYGZhy{}filename\PYGZhy{}/2G.fille \PYGZhy{}direct\PYGZhy{}1 \PYGZhy{}iodepth \PYG{l+m}{1} \PYGZhy{}thread \PYGZhy{}rw\PYGZhy{}randrw \PYGZhy{}rwmixread\PYGZhy{}70 \PYGZhy{}ioengine\PYGZhy{}psync \PYGZhy{}bs\PYGZhy{}16k \PYGZhy{}size\PYGZhy{}2G \PYGZhy{}numjobs\PYGZhy{}10 \PYGZhy{}runtime\PYGZhy{}60 \PYGZhy{}group\PYGZus{}reporting \PYGZhy{}name\PYGZhy{}mytest \PYGZhy{}ioscheduler\PYGZhy{}noop
\end{sphinxVerbatim}

\sphinxstylestrong{参数说明:}


\begin{savenotes}\sphinxattablestart
\centering
\begin{tabulary}{\linewidth}[t]{|T|T|}
\hline

filename-/dev/sdb1
&
测试文件名称,通常选择需要测试的盘的data目录。
\\
\hline
direct-1
&
测试过程绕过机器自带的buffer。使测试结果更真实。
\\
\hline
rw-randwrite
&
测试随机写的I/O
\\
\hline
rw-randrw
&
测试随机写和读的I/O
\\
\hline
bs-16k
&
单次io的块文件大小为16k
\\
\hline
bsrange-512-2048
&
同上,提定数据块的大小范围
\\
\hline
size-5g
&
本次的测试文件大小为5g,以每次4k的io进行测试。
\\
\hline
numjobs-30
&
本次的测试线程为30.
\\
\hline
runtime-1000
&
测试时间为1000秒,如果不写则一直将5g文件分4k每次写完为止。
\\
\hline
ioengine-psync
&
io引擎使用pync方式
\\
\hline
rwmixwrite-30
&
在混合读写的模式下,写占30\%
\\
\hline
group\_reporting
&
关于显示结果的,汇总每个进程的信息。
\\
\hline
lockmem-1g
&
只使用1g内存进行测试。
\\
\hline
zero\_buffers
&
用0初始化系统buffer。
\\
\hline
nrfiles-8
&
每个进程生成文件的数量。
\\
\hline
\end{tabulary}
\par
\sphinxattableend\end{savenotes}
\end{quote}


\section{进程线程}
\label{\detokenize{linux/shell:id3}}\begin{itemize}
\item {} 
查看线程

\begin{sphinxVerbatim}[commandchars=\\\{\}]
ps \PYGZhy{}Ledf \PYG{p}{\textbar{}} grep app1 \PYG{p}{\textbar{}} wc \PYGZhy{}l
\end{sphinxVerbatim}

\begin{sphinxVerbatim}[commandchars=\\\{\}]
\PYGZhy{}A   显示所有进程。
\PYGZhy{}d  显示所有进程,但不包括阶段作业领导者的进程。
\PYGZhy{}e  此参数的效果和指定\PYGZdq{}A\PYGZdq{}参数相同。
\PYGZhy{}f  显示UID,PPIP,C与STIME栏位。
\PYGZhy{}g  显示现行终端机下的所有进程,包括群组领导者的进程。
 h  不显示标题列。
\PYGZhy{}j  采用工作控制的格式显示进程状况。
\PYGZhy{}L  采用详细的格式来显示进程状况。
\end{sphinxVerbatim}

\end{itemize}


\section{查询编辑}
\label{\detokenize{linux/shell:id4}}\begin{quote}
\begin{itemize}
\item {} 
\sphinxstylestrong{Vim}

\noindent\sphinxincludegraphics{{vim}.png}

\item {} 
\sphinxstylestrong{sed(编辑)}
\begin{quote}

\begin{DUlineblock}{0em}
\item[] 以行为单位的文本编辑工具 sed可以直接修改档案。
\item[] 基本工作方式: sed {[}-nef{]} ‘{[}动作{]}’ {[}输入文本{]}
\end{DUlineblock}
\end{quote}

\end{itemize}
\begin{quote}

\begin{sphinxVerbatim}[commandchars=\\\{\}]
\PYG{o}{\PYGZhy{}}\PYG{n}{n}   \PYG{n}{安静模式}  \PYG{n}{一般sed用法中}\PYG{p}{,} \PYG{n}{来自stdin的数据一般会被列出到屏幕上}\PYG{p}{,} \PYG{n}{如果使用}\PYG{o}{\PYGZhy{}}\PYG{n}{n参数后}\PYG{p}{,} \PYG{n}{只有经过sed处理的那一行被列出来}\PYG{o}{.}
\PYG{o}{\PYGZhy{}}\PYG{n}{e}   \PYG{n}{多重编辑}  \PYG{n}{比如你同时又想删除某行}\PYG{p}{,} \PYG{n}{又想改变其他行}\PYG{p}{,} \PYG{n}{那么可以用} \PYG{n}{sed} \PYG{o}{\PYGZhy{}}\PYG{n}{e} \PYG{l+s+s1}{\PYGZsq{}}\PYG{l+s+s1}{1,5d}\PYG{l+s+s1}{\PYGZsq{}} \PYG{o}{\PYGZhy{}}\PYG{n}{e} \PYG{l+s+s1}{\PYGZsq{}}\PYG{l+s+s1}{s/abc/xxx/g}\PYG{l+s+s1}{\PYGZsq{}} \PYG{n}{filename}
\PYG{o}{\PYGZhy{}}\PYG{n}{f}   \PYG{n}{首先将sed的动作写在一个档案内}\PYG{p}{,} \PYG{n}{然后通过} \PYG{n}{sed} \PYG{o}{\PYGZhy{}}\PYG{n}{f} \PYG{n}{scriptfile} \PYG{n}{就可以直接执行} \PYG{n}{scriptfile} \PYG{n}{内的sed动作} \PYG{p}{(}\PYG{n}{没有实验成功}\PYG{p}{,} \PYG{n}{不推荐使用}\PYG{p}{)}
\PYG{o}{\PYGZhy{}}\PYG{n}{i}   \PYG{n}{直接编辑}\PYG{p}{,} \PYG{n}{这回就是真的改变文件中的内容了}\PYG{p}{,} \PYG{n}{别的都只是改变显示}\PYG{o}{.} \PYG{p}{(}\PYG{n}{不推荐使用}\PYG{p}{)}
\PYG{n}{动作}\PYG{p}{:}
\PYG{n}{a} \PYG{n}{新增}    \PYG{n}{后面可以接字符串}\PYG{p}{,} \PYG{n}{而这个字符串会在新的一行出现}\PYG{o}{.} \PYG{p}{(}\PYG{n}{下一行}\PYG{p}{)}
\PYG{n}{c} \PYG{n}{取代}    \PYG{n}{后面的字符串}\PYG{p}{,} \PYG{n}{这些字符串可以取代} \PYG{n}{n1}\PYG{p}{,}\PYG{n}{n2之间的行}
\PYG{n}{d} \PYG{n}{删除}    \PYG{n}{后面不接任何东西}
\PYG{n}{i} \PYG{n}{插入}    \PYG{n}{后面的字符串}\PYG{p}{,} \PYG{n}{会在上一行出现}
\PYG{n}{p} \PYG{n}{打印}    \PYG{n}{将选择的资料列出}\PYG{p}{,} \PYG{n}{通常和} \PYG{n}{sed} \PYG{o}{\PYGZhy{}}\PYG{n}{n} \PYG{n}{一起运作} \PYG{n}{sed} \PYG{o}{\PYGZhy{}}\PYG{n}{n} \PYG{l+s+s1}{\PYGZsq{}}\PYG{l+s+s1}{3p}\PYG{l+s+s1}{\PYGZsq{}} \PYG{n}{只打印第3行}
\PYG{n}{s} \PYG{n}{取代}    \PYG{n}{类似vi中的取代}\PYG{p}{,} \PYG{l+m+mi}{1}\PYG{p}{,}\PYG{l+m+mi}{20}\PYG{n}{s}\PYG{o}{/}\PYG{n}{old}\PYG{o}{/}\PYG{n}{new}\PYG{o}{/}\PYG{n}{g}
\end{sphinxVerbatim}

\sphinxstylestrong{举例:}

\begin{sphinxVerbatim}[commandchars=\\\{\}]
\PYG{c+c1}{\PYGZsh{}删除 abc 档案里的第一行, 注意, 这时会显示除了第一行之外的所有行, 因为第一行已经被删除了(实际文件并没有被删除,而只是显示的时候被删除了)}
sed \PYG{l+s+s1}{\PYGZsq{}1d\PYGZsq{}} abc

\PYG{c+c1}{\PYGZsh{}什么内容也不显示, 因为经过sed处理的行, 是个删除操作, 所以不现实.}
sed \PYGZhy{}n \PYG{l+s+s1}{\PYGZsq{}1d\PYGZsq{}} abc

\PYG{c+c1}{\PYGZsh{}abc 删除abc中从第二行到最后一行所有的内容, 注意, \PYGZdl{}符号正则表达式中表示行末尾, 但是这里并没有说那行末尾, 就会指最后一行末尾, \PYGZca{}开头, 如果没有指定哪行开头, 那么就是第一行开头}
sed \PYG{l+s+s1}{\PYGZsq{}2,\PYGZdl{}d\PYGZsq{}}

只删除了最后一行, 因为并没有指定是那行末尾, 就认为是最后一行末尾
sed \PYG{l+s+s1}{\PYGZsq{}\PYGZdl{}d\PYGZsq{}} abc

\PYG{c+c1}{\PYGZsh{}abc 文件中所有带 test 的行, 全部删除}
sed \PYG{l+s+s1}{\PYGZsq{}/test/d\PYGZsq{}}

\PYG{c+c1}{\PYGZsh{}abc 将 RRRRRRR 追加到所有的带 test 行的下一行 也有可能通过行 sed \PYGZsq{}1,5c RRRRRRR\PYGZsq{} abc}
sed \PYG{l+s+s1}{\PYGZsq{}/test/a RRRRRRR\PYGZsq{}}

\PYG{c+c1}{\PYGZsh{}abc 将 RRRRRRR 替换所有带 test 的行, 当然, 这里也可以是通过行来进行替换, 比如 sed \PYGZsq{}1,5c RRRRRRR\PYGZsq{} abc}
sed \PYG{l+s+s1}{\PYGZsq{}/test/c RRRRRRR\PYGZsq{}}
\end{sphinxVerbatim}
\end{quote}
\begin{itemize}
\item {} 
\sphinxstylestrong{awk(分析\&处理)}

awk ‘条件类型1\{动作1\}条件类型2\{动作2\}’ filename,

\begin{sphinxVerbatim}[commandchars=\\\{\}]
awk的处理流程是:
  1. 读第一行, 将第一行资料填入变量 \PYGZdl{}0, \PYGZdl{}1... 等变量中
  2. 依据条件限制, 执行动作
  3. 接下来执行下一行
所以, AWK一次处理是一行, 而一次中处理的最小单位是一个区域
另外还有3个变量, NF: 每一行处理的字段数, NR 目前处理到第几行 FS 目前的分隔符
逻辑判断 \PYGZgt{} \PYGZlt{} \PYGZgt{}= \PYGZlt{}= == !== , 赋值直接使用=
\end{sphinxVerbatim}

\sphinxstylestrong{举例:}

\begin{sphinxVerbatim}[commandchars=\\\{\}]
last \PYGZhy{}n \PYG{l+m}{5} \PYG{p}{\textbar{}} awk \PYG{l+s+s1}{\PYGZsq{}\PYGZob{}print \PYGZdl{}1 \PYGZdq{}\PYGZbs{}t\PYGZdq{} \PYGZdl{}3\PYGZcb{}\PYGZsq{}}
\PYG{c+c1}{\PYGZsh{}这里大括号内\PYGZdl{}1\PYGZdq{}\PYGZbs{}t\PYGZdq{}\PYGZdl{}3 之间不加空格也可以, 不过最好还是加上个空格,}
\PYG{c+c1}{\PYGZsh{}另外注意\PYGZdq{}\PYGZbs{}t\PYGZdq{}是有双引号的, 因为本身这些内容都在单引号内}
\PYG{c+c1}{\PYGZsh{}\PYGZdl{}0 代表整行 \PYGZdl{}1代表第一个区域, 依此类推}

cat /etc/passwd \PYG{p}{\textbar{}} awk \PYG{l+s+s1}{\PYGZsq{}\PYGZob{}FS=\PYGZdq{}:\PYGZdq{}\PYGZcb{} \PYGZdl{}3\PYGZlt{}10 \PYGZob{}print \PYGZdl{}1 \PYGZdq{}\PYGZbs{}t\PYGZdq{} \PYGZdl{}3\PYGZcb{}\PYGZsq{}}
\PYG{c+c1}{\PYGZsh{}首先定义分隔符为:, 然后判断, 注意看, 判断没有写在\PYGZob{}\PYGZcb{}中, 然后执行动作, FS=\PYGZdq{}:\PYGZdq{}这是一个动作, 赋值动作, 不是一个判断, 所以不写在\PYGZob{}\PYGZcb{}中}
\PYG{c+c1}{\PYGZsh{}BEGIN END , 给程序员一个初始化和收尾的工作, BEGIN之后列出的操作在\PYGZob{}\PYGZcb{}内将在awk开始扫描输入之前执行, 而END\PYGZob{}\PYGZcb{}内的操作, 将在扫描完输入文件后执行.}

awk \PYG{l+s+s1}{\PYGZsq{}/test/ \PYGZob{}print NR\PYGZcb{}\PYGZsq{}} abc \PYG{c+c1}{\PYGZsh{}将带有test的行的行号打印出来, 注意//之间可以使用正则表达式}
\PYG{c+c1}{\PYGZsh{}awk \PYGZob{}\PYGZcb{}内, 可以使用 if else ,for(i=0;i\PYGZlt{}10;i++), i=1 while(i\PYGZlt{}NF)}
\end{sphinxVerbatim}

\item {} 
\sphinxstylestrong{grep(截取)}

\begin{sphinxVerbatim}[commandchars=\\\{\}]
\PYG{o}{\PYGZhy{}}\PYG{n}{c}    \PYG{n}{只输出匹配的行}
\PYG{o}{\PYGZhy{}}\PYG{n}{I}    \PYG{n}{不区分大小写}
\PYG{o}{\PYGZhy{}}\PYG{n}{h}    \PYG{n}{查询多文件时不显示文件名}
\PYG{o}{\PYGZhy{}}\PYG{n}{l}    \PYG{n}{查询多文件时}\PYG{p}{,} \PYG{n}{只输出包含匹配字符的文件名}
\PYG{o}{\PYGZhy{}}\PYG{n}{n}    \PYG{n}{显示匹配的行号及行}
\PYG{o}{\PYGZhy{}}\PYG{n}{v}    \PYG{n}{显示不包含匹配文本的所有行}\PYG{p}{(}\PYG{n}{我经常用除去grep本身}\PYG{p}{)}
\end{sphinxVerbatim}

\end{itemize}
\end{quote}


\chapter{Mysql}
\label{\detokenize{linux/mysql:mysql}}\label{\detokenize{linux/mysql::doc}}\begin{itemize}
\item {} 
mysql 5.7 初始化

\begin{sphinxVerbatim}[commandchars=\\\{\}]
\PYG{c+c1}{\PYGZsh{}1.安装完成后默认随机密码}
\PYG{c+c1}{\PYGZsh{}my.cnf 文件 添加 skip\PYGZhy{}grant\PYGZhy{}tables 参数,安全模式启动。}
update mysql.user \PYG{n+nb}{set} \PYG{n+nv}{authentication\PYGZus{}string}\PYG{o}{=}\PYG{l+s+s1}{\PYGZsq{}\PYGZsq{}} where \PYG{n+nv}{user}\PYG{o}{=}\PYG{l+s+s1}{\PYGZsq{}root\PYGZsq{}}\PYG{p}{;} //设置空密码

\PYG{c+c1}{\PYGZsh{}2.密码安全策略:需要更改密码才能使用,密码复杂的策略}
\PYG{c+c1}{\PYGZsh{}my.cnf 文件 添加 validate\PYGZus{}password=off,default\PYGZus{}password\PYGZus{}lifetime=0 关闭密码复杂度要求及过期时间。}
\PYG{o}{[}mysqld\PYG{o}{]}
\PYG{n+nv}{default\PYGZus{}password\PYGZus{}lifetime}\PYG{o}{=}\PYG{l+m}{0}
\PYG{n+nv}{validate\PYGZus{}password}\PYG{o}{=}off
\PYG{c+c1}{\PYGZsh{}skip\PYGZhy{}grant\PYGZhy{}tables}
\end{sphinxVerbatim}

\item {} 
创建\&更改用户密码

\begin{sphinxVerbatim}[commandchars=\\\{\}]
\PYG{k+kt}{set} \PYG{n}{password} \PYG{k}{for} \PYG{n}{root}\PYG{o}{@}\PYG{l+s+s1}{\PYGZsq{}\PYGZpc{}\PYGZsq{}}\PYG{o}{=}\PYG{n+nf}{password}\PYG{p}{(}\PYG{l+s+s1}{\PYGZsq{}1234\PYGZhy{}abcd\PYGZsq{}}\PYG{p}{)}\PYG{p}{;}
\PYG{k}{create} \PYG{n}{user} \PYG{l+s+s1}{\PYGZsq{}root\PYGZsq{}}\PYG{o}{@}\PYG{l+s+s1}{\PYGZsq{}\PYGZpc{}\PYGZsq{}} \PYG{n}{identified} \PYG{k}{by} \PYG{l+s+s1}{\PYGZsq{}123456\PYGZsq{}}\PYG{p}{;}
\PYG{k}{grant} \PYG{k}{all} \PYG{n}{privileges} \PYG{k}{on} \PYG{o}{*}\PYG{p}{.}\PYG{o}{*} \PYG{k}{to} \PYG{l+s+s1}{\PYGZsq{}root\PYGZsq{}}\PYG{o}{@}\PYG{l+s+s1}{\PYGZsq{}\PYGZpc{}\PYGZsq{}}\PYG{p}{;}
\end{sphinxVerbatim}

\item {} 
mysql 时区问题

\begin{sphinxVerbatim}[commandchars=\\\{\}]
问题:
Django 2.0 ORM操作CONVERT\PYGZus{}TZ中传递的是时区位置,如mysql数据中无对应时区信息,将返回NULL
如:
SELECT CONVERT\PYGZus{}TZ(\PYGZsq{}2004\PYGZhy{}01\PYGZhy{}01 12:00:00\PYGZsq{},\PYGZsq{}GMT\PYGZsq{},\PYGZsq{}MET\PYGZsq{});,查询结果默认可能为null
SELECT CONVERT\PYGZus{}TZ(\PYGZsq{}2004\PYGZhy{}01\PYGZhy{}01 12:00:00\PYGZsq{},\PYGZsq{}+00:00\PYGZsq{},\PYGZsq{}+10:00\PYGZsq{}); 查询结果正常

解决方法:
CONVERT\PYGZus{}TZ(dt,from\PYGZus{}tz,to\PYGZus{}tz)转换datetime值dt,从 from\PYGZus{}tz 由给定转到 to\PYGZus{}tz 时区给出的时区,并返回的结果值。 如果参数无效该函数返回NULL。
mysql\PYGZus{}tzinfo\PYGZus{}to\PYGZus{}sql  /usr/share/zoneinfo \textbar{} mysql mysql  把系统的时区信息导入
\end{sphinxVerbatim}

\end{itemize}


\chapter{Kubernetes}
\label{\detokenize{linux/kubernetes:kubernetes}}\label{\detokenize{linux/kubernetes::doc}}\begin{itemize}
\item {} 
安装

\begin{sphinxVerbatim}[commandchars=\\\{\}]
cat \PYG{l+s}{\PYGZlt{}\PYGZlt{}EOF \PYGZgt{} /etc/yum.repos.d/kubernetes.repo}
\PYG{l+s}{[kubernetes]}
\PYG{l+s}{name=Kubernetes}
\PYG{l+s}{baseurl=https://packages.cloud.google.com/yum/repos/kubernetes\PYGZhy{}el7\PYGZhy{}x86\PYGZus{}64}
\PYG{l+s}{enabled=1}
\PYG{l+s}{gpgcheck=1}
\PYG{l+s}{repo\PYGZus{}gpgcheck=1}
\PYG{l+s}{gpgkey=https://packages.cloud.google.com/yum/doc/yum\PYGZhy{}key.gpg https://packages.cloud.google.com/yum/doc/rpm\PYGZhy{}package\PYGZhy{}key.gpg}
\PYG{l+s}{exclude=kube*}
\PYG{l+s}{EOF}

yum install \PYGZhy{}y kubelet kubeadm kubectl \PYGZhy{}\PYGZhy{}disableexcludes\PYG{o}{=}kubernetes
\end{sphinxVerbatim}

\item {} 
kubeadm init

\end{itemize}


\chapter{Vmware}
\label{\detokenize{vmware/index:vmware}}\label{\detokenize{vmware/index::doc}}

\section{PowerCLI}
\label{\detokenize{vmware/powercli:powercli}}\label{\detokenize{vmware/powercli::doc}}\begin{itemize}
\item {} 
\sphinxstylestrong{PowerCLI安装}

\begin{sphinxVerbatim}[commandchars=\\\{\}]
\PYG{c}{\PYGZsh{}PowerCLI installation with admin rights:}
\PYG{n+nb}{Install\PYGZhy{}Module} \PYG{n}{VMware}\PYG{p}{.}\PYG{n}{PowerCLI}

\PYG{c}{\PYGZsh{}Use the \PYGZhy{}AllowClobber when you get: A command with the name \PYGZsq{}Export\PYGZhy{}VM\PYGZsq{} is already available on this system.}
\PYG{n+nb}{Install\PYGZhy{}Module} \PYG{n}{VMware}\PYG{p}{.}\PYG{n}{PowerCLI} \PYG{n}{\PYGZhy{}AllowClobber}

\PYG{c}{\PYGZsh{}Installation of PowerCLI without admin rights:}
\PYG{n+nb}{Install\PYGZhy{}Module} \PYG{n}{VMware}\PYG{p}{.}\PYG{n}{PowerCLI} \PYG{n}{\PYGZhy{}Scope} \PYG{n}{CurrentUser}

\PYG{c}{\PYGZsh{}These modules are installed in the \PYGZpc{}homepath\PYGZpc{}\PYGZbs{}Documents\PYGZbs{}WindowsPowerShell\PYGZbs{}Modules}
\end{sphinxVerbatim}

\item {} 
\sphinxstylestrong{添加磁盘}

\begin{sphinxVerbatim}[commandchars=\\\{\}]
\PYG{n+nb}{Connect\PYGZhy{}VIServer} \PYG{n}{\PYGZhy{}Server} \PYG{n}{192}\PYG{p}{.}\PYG{n}{168}\PYG{p}{.}\PYG{n}{101}\PYG{p}{.}\PYG{n}{100} \PYG{n}{\PYGZhy{}Protocol} \PYG{n}{https} \PYG{n}{\PYGZhy{}User} \PYG{l+s+s1}{\PYGZsq{}administrator\PYGZsq{}} \PYG{n}{\PYGZhy{}Password} \PYG{l+s+s1}{\PYGZsq{}1234\PYGZhy{}abcd\PYGZsq{}}\PYG{n}{\PYGZhy{}Force}
\PYG{n+nv}{\PYGZdl{}vm} \PYG{p}{=} \PYG{n+nb}{Get\PYGZhy{}VM} \PYG{n}{wx}\PYG{n}{\PYGZhy{}tky}\PYG{n}{\PYGZhy{}ops}\PYG{p}{\PYGZhy{}}\PYG{n}{012}
\PYG{n+nv}{\PYGZdl{}vm} \PYG{p}{\textbar{}} \PYG{n+nb}{New\PYGZhy{}HardDisk} \PYG{n}{\PYGZhy{}CapacityGB} \PYG{n}{200} \PYG{n}{\PYGZhy{}Persistence} \PYG{n}{persistent}
\end{sphinxVerbatim}

\item {} 
\sphinxstylestrong{获取虚拟机创建时间}

\begin{sphinxVerbatim}[commandchars=\\\{\}]
\PYG{n+nb}{Connect\PYGZhy{}VIServer} \PYG{n}{\PYGZhy{}Server} \PYG{n}{192}\PYG{p}{.}\PYG{n}{168}\PYG{p}{.}\PYG{n}{101}\PYG{p}{.}\PYG{n}{100} \PYG{n}{\PYGZhy{}Protocol} \PYG{n}{https} \PYG{n}{\PYGZhy{}User} \PYG{l+s+s1}{\PYGZsq{}administrator\PYGZsq{}} \PYG{n}{\PYGZhy{}Password} \PYG{l+s+s1}{\PYGZsq{}1234\PYGZhy{}abcd\PYGZsq{}}\PYG{n}{\PYGZhy{}Force}
\PYG{n+nv}{\PYGZdl{}vms}\PYG{p}{=}\PYG{n+nb}{Get\PYGZhy{}VM}
\PYG{n+nv}{\PYGZdl{}Report} \PYG{p}{=} \PYG{p}{@}\PYG{p}{(}\PYG{p}{)}
\PYG{k}{foreach} \PYG{p}{(}\PYG{n+nv}{\PYGZdl{}vm} \PYG{k}{in} \PYG{n+nv}{\PYGZdl{}vms}\PYG{p}{)}
\PYG{p}{\PYGZob{}} \PYG{p}{,}
\PYG{n+nv}{\PYGZdl{}Reportobj}\PYG{p}{=}\PYG{l+s+s2}{\PYGZdq{}}\PYG{l+s+s2}{\PYGZdq{}} \PYG{p}{\textbar{}} \PYG{n+nb}{Select }\PYG{l+s+s2}{\PYGZdq{}}\PYG{l+s+s2}{vmname}\PYG{l+s+s2}{\PYGZdq{}}\PYG{p}{,}  \PYG{l+s+s2}{\PYGZdq{}}\PYG{l+s+s2}{vmcreatetime}\PYG{l+s+s2}{\PYGZdq{}}
\PYG{n+nv}{\PYGZdl{}Reportobj}\PYG{p}{.}\PYG{n}{vmname}\PYG{p}{=}\PYG{n+nv}{\PYGZdl{}vm}\PYG{p}{.}\PYG{n}{Name}
\PYG{n+nv}{\PYGZdl{}Reportobj}\PYG{p}{.}\PYG{n}{vmcreatetime}\PYG{p}{=}\PYG{n+nv}{\PYGZdl{}vm}\PYG{p}{.}\PYG{n}{ExtensionData}\PYG{p}{.}\PYG{n}{Config}\PYG{p}{.}\PYG{n}{CreateDate}
\PYG{n+nv}{\PYGZdl{}Report}\PYG{p}{+}\PYG{p}{=}\PYG{n+nv}{\PYGZdl{}Reportobj}
\PYG{p}{\PYGZcb{}}
\PYG{n+nv}{\PYGZdl{}Report} \PYG{p}{\textbar{}} \PYG{n+nb}{Export\PYGZhy{}Csv} \PYG{n}{\PYGZhy{}NoTypeInformation} \PYG{n}{\PYGZhy{}Encoding} \PYG{n}{UTF8} \PYG{n}{\PYGZhy{}path} \PYG{n}{vm}\PYG{n}{\PYGZhy{}Info}\PYG{p}{.}\PYG{n}{csv}
\end{sphinxVerbatim}

\item {} 
\sphinxstylestrong{获取esxi主机信息}

\begin{sphinxVerbatim}[commandchars=\\\{\}]
\PYG{n+nb}{Connect\PYGZhy{}VIServer} \PYG{n}{\PYGZhy{}Server} \PYG{n}{192}\PYG{p}{.}\PYG{n}{168}\PYG{p}{.}\PYG{n}{101}\PYG{p}{.}\PYG{n}{100} \PYG{n}{\PYGZhy{}Protocol} \PYG{n}{https} \PYG{n}{\PYGZhy{}User} \PYG{l+s+s1}{\PYGZsq{}administrator\PYGZsq{}} \PYG{n}{\PYGZhy{}Password} \PYG{l+s+s1}{\PYGZsq{}1234\PYGZhy{}abcd\PYGZsq{}}\PYG{n}{\PYGZhy{}Force}
\PYG{c}{\PYGZsh{}\PYGZdl{}respool = Get\PYGZhy{}ResourcePool dev}
\PYG{c}{\PYGZsh{}Get\PYGZhy{}VM \PYGZhy{}Location \PYGZdl{}respool \PYGZsh{}\textbar{} Select\PYGZhy{}Object Name,NumCpu,MemoryGB,PowerState,VMHost}
\PYG{n+nb}{get\PYGZhy{}vm} \PYG{p}{\textbar{}} \PYG{n+nb}{Where\PYGZhy{}Object}\PYG{p}{\PYGZob{}}\PYG{n+nv}{\PYGZdl{}\PYGZus{}}\PYG{p}{.}\PYG{n}{powerstate} \PYG{o}{\PYGZhy{}eq} \PYG{l+s+s1}{\PYGZsq{}PoweredOn\PYGZsq{}}\PYG{p}{\PYGZcb{}} \PYG{p}{\textbar{}}\PYG{n+nb}{Measure\PYGZhy{}Object} \PYG{c}{\PYGZsh{}统计在线虚拟机数量}
\PYG{n+nv}{\PYGZdl{}Report} \PYG{p}{=} \PYG{p}{@}\PYG{p}{(}\PYG{p}{)}
\PYG{n+nv}{\PYGZdl{}ESXHosts} \PYG{p}{=} \PYG{n+nb}{Get\PYGZhy{}VMHost}
\PYG{k}{ForEach} \PYG{p}{(}\PYG{n+nv}{\PYGZdl{}ESXHost} \PYG{k}{in} \PYG{n+nv}{\PYGZdl{}ESXHosts}\PYG{p}{)}
\PYG{p}{\PYGZob{}}
\PYG{n+nv}{\PYGZdl{}ReportObj} \PYG{p}{=} \PYG{l+s+s2}{\PYGZdq{}}\PYG{l+s+s2}{\PYGZdq{}} \PYG{p}{\textbar{}} \PYG{n+nb}{Select }\PYG{l+s+s2}{\PYGZdq{}}\PYG{l+s+s2}{ESXi 主机名}\PYG{l+s+s2}{\PYGZdq{}}\PYG{p}{,}  \PYG{l+s+s2}{\PYGZdq{}}\PYG{l+s+s2}{所属群集}\PYG{l+s+s2}{\PYGZdq{}}\PYG{p}{,} \PYG{l+s+s2}{\PYGZdq{}}\PYG{l+s+s2}{VMKernel IP}\PYG{l+s+s2}{\PYGZdq{}}\PYG{p}{,} \PYG{l+s+s2}{\PYGZdq{}}\PYG{l+s+s2}{ESXi 全版本}\PYG{l+s+s2}{\PYGZdq{}}\PYG{p}{,} \PYG{l+s+s2}{\PYGZdq{}}\PYG{l+s+s2}{ESXi 主版本}\PYG{l+s+s2}{\PYGZdq{}}\PYG{p}{,} \PYG{l+s+s2}{\PYGZdq{}}\PYG{l+s+s2}{ESXi 子版本}\PYG{l+s+s2}{\PYGZdq{}}\PYG{p}{,} \PYG{l+s+s2}{\PYGZdq{}}\PYG{l+s+s2}{许可证序号}\PYG{l+s+s2}{\PYGZdq{}}\PYG{p}{,} \PYG{l+s+s2}{\PYGZdq{}}\PYG{l+s+s2}{许可证版本}\PYG{l+s+s2}{\PYGZdq{}}\PYG{p}{,} \PYG{l+s+s2}{\PYGZdq{}}\PYG{l+s+s2}{UUID}\PYG{l+s+s2}{\PYGZdq{}}\PYG{p}{,} \PYG{l+s+s2}{\PYGZdq{}}\PYG{l+s+s2}{制造商}\PYG{l+s+s2}{\PYGZdq{}}\PYG{p}{,} \PYG{l+s+s2}{\PYGZdq{}}\PYG{l+s+s2}{型号}\PYG{l+s+s2}{\PYGZdq{}}\PYG{p}{,} \PYG{l+s+s2}{\PYGZdq{}}\PYG{l+s+s2}{BIOS 版本}\PYG{l+s+s2}{\PYGZdq{}}\PYG{p}{,} \PYG{l+s+s2}{\PYGZdq{}}\PYG{l+s+s2}{BIOS 发布日期}\PYG{l+s+s2}{\PYGZdq{}}\PYG{p}{,} \PYG{l+s+s2}{\PYGZdq{}}\PYG{l+s+s2}{设备序列号}\PYG{l+s+s2}{\PYGZdq{}}\PYG{p}{,} \PYG{l+s+s2}{\PYGZdq{}}\PYG{l+s+s2}{电源状态}\PYG{l+s+s2}{\PYGZdq{}}\PYG{p}{,} \PYG{l+s+s2}{\PYGZdq{}}\PYG{l+s+s2}{连接状态}\PYG{l+s+s2}{\PYGZdq{}}\PYG{p}{,} \PYG{l+s+s2}{\PYGZdq{}}\PYG{l+s+s2}{最后一次启动时间}\PYG{l+s+s2}{\PYGZdq{}}\PYG{p}{,} \PYG{l+s+s2}{\PYGZdq{}}\PYG{l+s+s2}{vMotion 启用状态}\PYG{l+s+s2}{\PYGZdq{}}\PYG{p}{,} \PYG{l+s+s2}{\PYGZdq{}}\PYG{l+s+s2}{FaultTolerance 启用状态}\PYG{l+s+s2}{\PYGZdq{}}\PYG{p}{,} \PYG{l+s+s2}{\PYGZdq{}}\PYG{l+s+s2}{CPU 型号}\PYG{l+s+s2}{\PYGZdq{}}\PYG{p}{,} \PYG{l+s+s2}{\PYGZdq{}}\PYG{l+s+s2}{CPU 插槽数}\PYG{l+s+s2}{\PYGZdq{}}\PYG{p}{,} \PYG{l+s+s2}{\PYGZdq{}}\PYG{l+s+s2}{每 CPU 内核数}\PYG{l+s+s2}{\PYGZdq{}}\PYG{p}{,} \PYG{l+s+s2}{\PYGZdq{}}\PYG{l+s+s2}{物理 CPU 内核数}\PYG{l+s+s2}{\PYGZdq{}}\PYG{p}{,} \PYG{l+s+s2}{\PYGZdq{}}\PYG{l+s+s2}{逻辑 CPU 内核数}\PYG{l+s+s2}{\PYGZdq{}}\PYG{p}{,} \PYG{l+s+s2}{\PYGZdq{}}\PYG{l+s+s2}{超线程启用状态}\PYG{l+s+s2}{\PYGZdq{}}\PYG{p}{,} \PYG{l+s+s2}{\PYGZdq{}}\PYG{l+s+s2}{每 CPU 速度(MHz)}\PYG{l+s+s2}{\PYGZdq{}}\PYG{p}{,} \PYG{l+s+s2}{\PYGZdq{}}\PYG{l+s+s2}{CPU 总速度(MHz)}\PYG{l+s+s2}{\PYGZdq{}}\PYG{p}{,} \PYG{l+s+s2}{\PYGZdq{}}\PYG{l+s+s2}{CPU 已用速度(MHz)}\PYG{l+s+s2}{\PYGZdq{}}\PYG{p}{,} \PYG{l+s+s2}{\PYGZdq{}}\PYG{l+s+s2}{内存总容量 GB}\PYG{l+s+s2}{\PYGZdq{}}\PYG{p}{,} \PYG{l+s+s2}{\PYGZdq{}}\PYG{l+s+s2}{内存使用量 GB}\PYG{l+s+s2}{\PYGZdq{}}\PYG{p}{,} \PYG{l+s+s2}{\PYGZdq{}}\PYG{l+s+s2}{网卡数}\PYG{l+s+s2}{\PYGZdq{}}\PYG{p}{,} \PYG{l+s+s2}{\PYGZdq{}}\PYG{l+s+s2}{HBA 卡数}\PYG{l+s+s2}{\PYGZdq{}}\PYG{p}{,}\PYG{l+s+s2}{\PYGZdq{}}\PYG{l+s+s2}{备注}\PYG{l+s+s2}{\PYGZdq{}}
\PYG{n+nv}{\PYGZdl{}ESXHost\PYGZus{}Temp} \PYG{p}{=} \PYG{p}{(}\PYG{n+nv}{\PYGZdl{}ESXHost} \PYG{p}{\textbar{}} \PYG{n+nb}{Get\PYGZhy{}View}\PYG{p}{)}
\PYG{n+nv}{\PYGZdl{}ESXHost\PYGZus{}SerialNumber}\PYG{p}{=}\PYG{n+nb}{Get\PYGZhy{}EsxCli} \PYG{n}{\PYGZhy{}VMHost} \PYG{n+nv}{\PYGZdl{}ESXHost}
\PYG{n+nv}{\PYGZdl{}ReportObj}\PYG{p}{.}\PYG{l+s+s2}{\PYGZdq{}}\PYG{l+s+s2}{ESXi 主机名}\PYG{l+s+s2}{\PYGZdq{}} \PYG{p}{=} \PYG{n+nv}{\PYGZdl{}ESXHost}\PYG{p}{.}\PYG{n}{Name}
\PYG{n+nv}{\PYGZdl{}ReportObj}\PYG{p}{.}\PYG{l+s+s2}{\PYGZdq{}}\PYG{l+s+s2}{所属群集}\PYG{l+s+s2}{\PYGZdq{}} \PYG{p}{=} \PYG{n+nv}{\PYGZdl{}ESXHost}\PYG{p}{.}\PYG{n}{Parent}
\PYG{n+nv}{\PYGZdl{}ReportObj}\PYG{p}{.}\PYG{l+s+s2}{\PYGZdq{}}\PYG{l+s+s2}{ESXi 主版本}\PYG{l+s+s2}{\PYGZdq{}} \PYG{p}{=} \PYG{n+nv}{\PYGZdl{}ESXHost}\PYG{p}{.}\PYG{n}{Version}
\PYG{n+nv}{\PYGZdl{}ReportObj}\PYG{p}{.}\PYG{l+s+s2}{\PYGZdq{}}\PYG{l+s+s2}{ESXi 子版本}\PYG{l+s+s2}{\PYGZdq{}} \PYG{p}{=} \PYG{n+nv}{\PYGZdl{}ESXHost}\PYG{p}{.}\PYG{n}{Build}
\PYG{n+nv}{\PYGZdl{}ReportObj}\PYG{p}{.}\PYG{l+s+s2}{\PYGZdq{}}\PYG{l+s+s2}{许可证序号}\PYG{l+s+s2}{\PYGZdq{}} \PYG{p}{=} \PYG{n+nv}{\PYGZdl{}ESXHost}\PYG{p}{.}\PYG{n}{LicenseKey}
\PYG{n+nv}{\PYGZdl{}ReportObj}\PYG{p}{.}\PYG{l+s+s2}{\PYGZdq{}}\PYG{l+s+s2}{制造商}\PYG{l+s+s2}{\PYGZdq{}} \PYG{p}{=} \PYG{n+nv}{\PYGZdl{}ESXHost}\PYG{p}{.}\PYG{n}{Manufacturer}
\PYG{n+nv}{\PYGZdl{}ReportObj}\PYG{p}{.}\PYG{l+s+s2}{\PYGZdq{}}\PYG{l+s+s2}{型号}\PYG{l+s+s2}{\PYGZdq{}} \PYG{p}{=} \PYG{n+nv}{\PYGZdl{}ESXHost}\PYG{p}{.}\PYG{n}{Model}
\PYG{n+nv}{\PYGZdl{}ReportObj}\PYG{p}{.}\PYG{l+s+s2}{\PYGZdq{}}\PYG{l+s+s2}{电源状态}\PYG{l+s+s2}{\PYGZdq{}} \PYG{p}{=} \PYG{n+nv}{\PYGZdl{}ESXHost}\PYG{p}{.}\PYG{n}{PowerState}
\PYG{n+nv}{\PYGZdl{}ReportObj}\PYG{p}{.}\PYG{l+s+s2}{\PYGZdq{}}\PYG{l+s+s2}{连接状态}\PYG{l+s+s2}{\PYGZdq{}} \PYG{p}{=} \PYG{n+nv}{\PYGZdl{}ESXHost}\PYG{p}{.}\PYG{n}{ConnectionState}
\PYG{n+nv}{\PYGZdl{}ReportObj}\PYG{p}{.}\PYG{l+s+s2}{\PYGZdq{}}\PYG{l+s+s2}{CPU 型号}\PYG{l+s+s2}{\PYGZdq{}} \PYG{p}{=} \PYG{n+nv}{\PYGZdl{}ESXHost}\PYG{p}{.}\PYG{n}{ProcessorType}
\PYG{n+nv}{\PYGZdl{}ReportObj}\PYG{p}{.}\PYG{l+s+s2}{\PYGZdq{}}\PYG{l+s+s2}{物理 CPU 内核数}\PYG{l+s+s2}{\PYGZdq{}} \PYG{p}{=} \PYG{n+nv}{\PYGZdl{}ESXHost}\PYG{p}{.}\PYG{n}{NumCpu}
\PYG{n+nv}{\PYGZdl{}ReportObj}\PYG{p}{.}\PYG{l+s+s2}{\PYGZdq{}}\PYG{l+s+s2}{超线程启用状态}\PYG{l+s+s2}{\PYGZdq{}} \PYG{p}{=} \PYG{n+nv}{\PYGZdl{}ESXHost}\PYG{p}{.}\PYG{n}{HyperthreadingActive}
\PYG{n+nv}{\PYGZdl{}ReportObj}\PYG{p}{.}\PYG{l+s+s2}{\PYGZdq{}}\PYG{l+s+s2}{CPU 总速度(MHz)}\PYG{l+s+s2}{\PYGZdq{}} \PYG{p}{=} \PYG{n+nv}{\PYGZdl{}ESXHost}\PYG{p}{.}\PYG{n}{CpuTotalMhz}
\PYG{n+nv}{\PYGZdl{}ReportObj}\PYG{p}{.}\PYG{l+s+s2}{\PYGZdq{}}\PYG{l+s+s2}{CPU 已用速度(MHz)}\PYG{l+s+s2}{\PYGZdq{}} \PYG{p}{=} \PYG{n+nv}{\PYGZdl{}ESXHost}\PYG{p}{.}\PYG{n}{CpuUsageMhz}
\PYG{n+nv}{\PYGZdl{}ReportObj}\PYG{p}{.}\PYG{l+s+s2}{\PYGZdq{}}\PYG{l+s+s2}{内存总容量 GB}\PYG{l+s+s2}{\PYGZdq{}} \PYG{p}{=} \PYG{n+no}{[math]}\PYG{p}{::}\PYG{n}{round}\PYG{p}{(}\PYG{n+nv}{\PYGZdl{}ESXHost}\PYG{p}{.}\PYG{n}{MemoryTotalGB}\PYG{p}{,} \PYG{n}{0}\PYG{p}{)}
\PYG{n+nv}{\PYGZdl{}ReportObj}\PYG{p}{.}\PYG{l+s+s2}{\PYGZdq{}}\PYG{l+s+s2}{内存使用量 GB}\PYG{l+s+s2}{\PYGZdq{}} \PYG{p}{=} \PYG{n+no}{[math]}\PYG{p}{::}\PYG{n}{round}\PYG{p}{(}\PYG{n+nv}{\PYGZdl{}ESXHost}\PYG{p}{.}\PYG{n}{MemoryUsageGB}\PYG{p}{,} \PYG{n}{0}\PYG{p}{)}
\PYG{n+nv}{\PYGZdl{}ReportObj}\PYG{p}{.}\PYG{l+s+s2}{\PYGZdq{}}\PYG{l+s+s2}{VMKernel IP}\PYG{l+s+s2}{\PYGZdq{}} \PYG{p}{=} \PYG{n+nv}{\PYGZdl{}ESXHost\PYGZus{}Temp}\PYG{p}{.}\PYG{n}{Config}\PYG{p}{.}\PYG{n}{Option} \PYG{p}{\textbar{}} \PYG{p}{?}\PYG{p}{\PYGZob{}}\PYG{n+nv}{\PYGZdl{}\PYGZus{}}\PYG{p}{.}\PYG{n}{Key} \PYG{o}{\PYGZhy{}like} \PYG{l+s+s2}{\PYGZdq{}}\PYG{l+s+s2}{Vpx.Vpxa.config.vpxa.hostIp}\PYG{l+s+s2}{\PYGZdq{}}\PYG{p}{\PYGZcb{}} \PYG{p}{\textbar{}} \PYG{p}{\PYGZpc{}} \PYG{p}{\PYGZob{}}\PYG{n+nv}{\PYGZdl{}\PYGZus{}}\PYG{p}{.}\PYG{n}{Value}\PYG{p}{\PYGZcb{}}
\PYG{n+nv}{\PYGZdl{}ReportObj}\PYG{p}{.}\PYG{l+s+s2}{\PYGZdq{}}\PYG{l+s+s2}{ESXi 全版本}\PYG{l+s+s2}{\PYGZdq{}} \PYG{p}{=} \PYG{n+nv}{\PYGZdl{}ESXHost\PYGZus{}Temp}\PYG{p}{.}\PYG{n}{Config}\PYG{p}{.}\PYG{n}{Product}\PYG{p}{.}\PYG{n}{FullName}
\PYG{n+nv}{\PYGZdl{}ReportObj}\PYG{p}{.}\PYG{l+s+s2}{\PYGZdq{}}\PYG{l+s+s2}{许可证版本}\PYG{l+s+s2}{\PYGZdq{}} \PYG{p}{=} \PYG{n+nv}{\PYGZdl{}ESXHost\PYGZus{}Temp}\PYG{p}{.}\PYG{n}{Config}\PYG{p}{.}\PYG{n}{Product}\PYG{p}{.}\PYG{n}{LicenseProductVersion}
\PYG{n+nv}{\PYGZdl{}ReportObj}\PYG{p}{.}\PYG{l+s+s2}{\PYGZdq{}}\PYG{l+s+s2}{UUID}\PYG{l+s+s2}{\PYGZdq{}} \PYG{p}{=} \PYG{n+nv}{\PYGZdl{}ESXHost\PYGZus{}Temp}\PYG{p}{.}\PYG{n}{Summary}\PYG{p}{.}\PYG{n}{Hardware}\PYG{p}{.}\PYG{n}{Uuid}
\PYG{n+nv}{\PYGZdl{}ReportObj}\PYG{p}{.}\PYG{l+s+s2}{\PYGZdq{}}\PYG{l+s+s2}{BIOS 版本}\PYG{l+s+s2}{\PYGZdq{}} \PYG{p}{=} \PYG{n+nv}{\PYGZdl{}ESXHost\PYGZus{}Temp}\PYG{p}{.}\PYG{n}{Hardware}\PYG{p}{.}\PYG{n}{BiosInfo}\PYG{p}{.}\PYG{n}{BiosVersion}
\PYG{n+nv}{\PYGZdl{}ReportObj}\PYG{p}{.}\PYG{l+s+s2}{\PYGZdq{}}\PYG{l+s+s2}{BIOS 发布日期}\PYG{l+s+s2}{\PYGZdq{}} \PYG{p}{=} \PYG{n+nv}{\PYGZdl{}ESXHost\PYGZus{}Temp}\PYG{p}{.}\PYG{n}{Hardware}\PYG{p}{.}\PYG{n}{BiosInfo}\PYG{p}{.}\PYG{n}{ReleaseDate}
\PYG{n+nv}{\PYGZdl{}ReportObj}\PYG{p}{.}\PYG{l+s+s2}{\PYGZdq{}}\PYG{l+s+s2}{最后一次启动时间}\PYG{l+s+s2}{\PYGZdq{}} \PYG{p}{=} \PYG{n+nv}{\PYGZdl{}ESXHost\PYGZus{}Temp}\PYG{p}{.}\PYG{n}{Summary}\PYG{p}{.}\PYG{n}{Runtime}\PYG{p}{.}\PYG{n}{BootTime}
\PYG{n+nv}{\PYGZdl{}ReportObj}\PYG{p}{.}\PYG{l+s+s2}{\PYGZdq{}}\PYG{l+s+s2}{vMotion 启用状态}\PYG{l+s+s2}{\PYGZdq{}} \PYG{p}{=} \PYG{n+nv}{\PYGZdl{}ESXHost\PYGZus{}Temp}\PYG{p}{.}\PYG{n}{Summary}\PYG{p}{.}\PYG{n}{Config}\PYG{p}{.}\PYG{n}{VmotionEnabled}
\PYG{n+nv}{\PYGZdl{}ReportObj}\PYG{p}{.}\PYG{l+s+s2}{\PYGZdq{}}\PYG{l+s+s2}{FaultTolerance 启用状态}\PYG{l+s+s2}{\PYGZdq{}} \PYG{p}{=} \PYG{n+nv}{\PYGZdl{}ESXHost\PYGZus{}Temp}\PYG{p}{.}\PYG{n}{Summary}\PYG{p}{.}\PYG{n}{Config}\PYG{p}{.}\PYG{n}{FaultToleranceEnabled}
\PYG{n+nv}{\PYGZdl{}ReportObj}\PYG{p}{.}\PYG{l+s+s2}{\PYGZdq{}}\PYG{l+s+s2}{CPU 插槽数}\PYG{l+s+s2}{\PYGZdq{}} \PYG{p}{=} \PYG{n+nv}{\PYGZdl{}ESXHost\PYGZus{}Temp}\PYG{p}{.}\PYG{n}{Summary}\PYG{p}{.}\PYG{n}{Hardware}\PYG{p}{.}\PYG{n}{NumCpuPkgs}
\PYG{n+nv}{\PYGZdl{}ReportObj}\PYG{p}{.}\PYG{l+s+s2}{\PYGZdq{}}\PYG{l+s+s2}{每 CPU 内核数}\PYG{l+s+s2}{\PYGZdq{}} \PYG{p}{=} \PYG{p}{(}\PYG{n+nv}{\PYGZdl{}ESXHost}\PYG{p}{.}\PYG{n}{NumCpu} \PYG{p}{/} \PYG{n+nv}{\PYGZdl{}ESXHost\PYGZus{}Temp}\PYG{p}{.}\PYG{n}{Summary}\PYG{p}{.}\PYG{n}{Hardware}\PYG{p}{.}\PYG{n}{NumCpuPkgs}\PYG{p}{)}
\PYG{n+nv}{\PYGZdl{}ReportObj}\PYG{p}{.}\PYG{l+s+s2}{\PYGZdq{}}\PYG{l+s+s2}{逻辑 CPU 内核数}\PYG{l+s+s2}{\PYGZdq{}} \PYG{p}{=} \PYG{n+nv}{\PYGZdl{}ESXHost\PYGZus{}Temp}\PYG{p}{.}\PYG{n}{Summary}\PYG{p}{.}\PYG{n}{Hardware}\PYG{p}{.}\PYG{n}{NumCpuThreads}
\PYG{n+nv}{\PYGZdl{}ReportObj}\PYG{p}{.}\PYG{l+s+s2}{\PYGZdq{}}\PYG{l+s+s2}{每 CPU 速度(MHz)}\PYG{l+s+s2}{\PYGZdq{}} \PYG{p}{=} \PYG{n+nv}{\PYGZdl{}ESXHost\PYGZus{}Temp}\PYG{p}{.}\PYG{n}{Summary}\PYG{p}{.}\PYG{n}{Hardware}\PYG{p}{.}\PYG{n}{CpuMhz}
\PYG{n+nv}{\PYGZdl{}ReportObj}\PYG{p}{.}\PYG{l+s+s2}{\PYGZdq{}}\PYG{l+s+s2}{网卡数}\PYG{l+s+s2}{\PYGZdq{}} \PYG{p}{=} \PYG{n+nv}{\PYGZdl{}ESXHost\PYGZus{}Temp}\PYG{p}{.}\PYG{n}{Summary}\PYG{p}{.}\PYG{n}{Hardware}\PYG{p}{.}\PYG{n}{NumNics}
\PYG{n+nv}{\PYGZdl{}ReportObj}\PYG{p}{.}\PYG{l+s+s2}{\PYGZdq{}}\PYG{l+s+s2}{HBA 卡数}\PYG{l+s+s2}{\PYGZdq{}} \PYG{p}{=} \PYG{n+nv}{\PYGZdl{}ESXHost\PYGZus{}Temp}\PYG{p}{.}\PYG{n}{Summary}\PYG{p}{.}\PYG{n}{Hardware}\PYG{p}{.}\PYG{n}{NumHBAs}
\PYG{n+nv}{\PYGZdl{}ReportObj}\PYG{p}{.}\PYG{l+s+s2}{\PYGZdq{}}\PYG{l+s+s2}{设备序列号}\PYG{l+s+s2}{\PYGZdq{}} \PYG{p}{=} \PYG{n+nv}{\PYGZdl{}ESXHost\PYGZus{}SerialNumber}\PYG{p}{.}\PYG{n}{hardware}\PYG{p}{.}\PYG{n}{platform}\PYG{p}{.}\PYG{n}{get}\PYG{p}{(}\PYG{p}{)}\PYG{p}{.}\PYG{n}{SerialNumber}
\PYG{n+nv}{\PYGZdl{}ReportObj}\PYG{p}{.}\PYG{l+s+s2}{\PYGZdq{}}\PYG{l+s+s2}{备注}\PYG{l+s+s2}{\PYGZdq{}} \PYG{p}{=} \PYG{l+s+s2}{\PYGZdq{}}\PYG{l+s+s2}{\PYGZdq{}}
\PYG{n+nv}{\PYGZdl{}Report} \PYG{p}{+}\PYG{p}{=} \PYG{n+nv}{\PYGZdl{}ReportObj}
\PYG{p}{\PYGZcb{}}
\PYG{n+nv}{\PYGZdl{}Report} \PYG{p}{\textbar{}} \PYG{n+nb}{Export\PYGZhy{}Csv} \PYG{n}{\PYGZhy{}NoTypeInformation} \PYG{n}{\PYGZhy{}Encoding} \PYG{n}{UTF8} \PYG{n}{\PYGZhy{}path} \PYG{n}{Esxi}\PYG{n}{\PYGZhy{}Host}\PYG{n}{\PYGZhy{}Info}\PYG{p}{.}\PYG{n}{csv}
\end{sphinxVerbatim}

\end{itemize}


\section{vcsa deploy}
\label{\detokenize{vmware/vcsa_deploy:vcsa-deploy}}\label{\detokenize{vmware/vcsa_deploy::doc}}\begin{itemize}
\item {} 
\sphinxstylestrong{install cli}

\begin{sphinxVerbatim}[commandchars=\\\{\}]
vcsa\PYGZhy{}deploy install \PYGZhy{}\PYGZhy{}no\PYGZhy{}ssl\PYGZhy{}certificate\PYGZhy{}verification \PYGZhy{}\PYGZhy{}accept\PYGZhy{}eula \PYGZhy{}\PYGZhy{}acknowledge\PYGZhy{}ceip C:\PYG{l+s+se}{\PYGZbs{}V}CSA\PYG{l+s+se}{\PYGZbs{}v}csa\PYGZhy{}cli\PYGZhy{}installer\PYG{l+s+se}{\PYGZbs{}V}CSA\PYGZus{}deploy.json
\end{sphinxVerbatim}

\item {} 
\sphinxstylestrong{embedded\_vCSA\_on\_ESXi.json}

\begin{sphinxVerbatim}[commandchars=\\\{\},numbers=left,firstnumber=1,stepnumber=1]
\PYG{p}{\PYGZob{}}
    \PYG{n+nt}{\PYGZdq{}\PYGZus{}\PYGZus{}version\PYGZdq{}}\PYG{p}{:} \PYG{l+s+s2}{\PYGZdq{}2.13.0\PYGZdq{}}\PYG{p}{,}
    \PYG{n+nt}{\PYGZdq{}\PYGZus{}\PYGZus{}comments\PYGZdq{}}\PYG{p}{:} \PYG{l+s+s2}{\PYGZdq{}Sample template to deploy a vCenter Server Appliance with an embedded Platform Services Controller on an ESXi host.\PYGZdq{}}\PYG{p}{,}
    \PYG{n+nt}{\PYGZdq{}new\PYGZus{}vcsa\PYGZdq{}}\PYG{p}{:} \PYG{p}{\PYGZob{}}
        \PYG{n+nt}{\PYGZdq{}esxi\PYGZdq{}}\PYG{p}{:} \PYG{p}{\PYGZob{}}
            \PYG{n+nt}{\PYGZdq{}hostname\PYGZdq{}}\PYG{p}{:} \PYG{l+s+s2}{\PYGZdq{}192.168.101.1\PYGZdq{}}\PYG{p}{,}
            \PYG{n+nt}{\PYGZdq{}username\PYGZdq{}}\PYG{p}{:} \PYG{l+s+s2}{\PYGZdq{}root\PYGZdq{}}\PYG{p}{,}
            \PYG{n+nt}{\PYGZdq{}password\PYGZdq{}}\PYG{p}{:} \PYG{l+s+s2}{\PYGZdq{}password\PYGZdq{}}\PYG{p}{,}
            \PYG{n+nt}{\PYGZdq{}deployment\PYGZus{}network\PYGZdq{}}\PYG{p}{:} \PYG{l+s+s2}{\PYGZdq{}DP101\PYGZdq{}}\PYG{p}{,}
            \PYG{n+nt}{\PYGZdq{}datastore\PYGZdq{}}\PYG{p}{:} \PYG{l+s+s2}{\PYGZdq{}vsanDatastore\PYGZdq{}}
        \PYG{p}{\PYGZcb{}}\PYG{p}{,}
         \PYG{n+nt}{\PYGZdq{}appliance\PYGZdq{}}\PYG{p}{:} \PYG{p}{\PYGZob{}}
            \PYG{n+nt}{\PYGZdq{}\PYGZus{}\PYGZus{}comments\PYGZdq{}}\PYG{p}{:} \PYG{p}{[}
                \PYG{l+s+s2}{\PYGZdq{}You must provide the \PYGZsq{}deployment\PYGZus{}option\PYGZsq{} key with a value, which will affect the VCSA\PYGZsq{}s configuration parameters, such as the VCSA\PYGZsq{}s number of vCPUs, the memory size, the storage size, and the maximum numbers of ESXi hosts and VMs which can be managed. For a list of acceptable values, run the supported deployment sizes help, i.e. vcsa\PYGZhy{}deploy \PYGZhy{}\PYGZhy{}supported\PYGZhy{}deployment\PYGZhy{}sizes\PYGZdq{}}
            \PYG{p}{]}\PYG{p}{,}
            \PYG{n+nt}{\PYGZdq{}thin\PYGZus{}disk\PYGZus{}mode\PYGZdq{}}\PYG{p}{:} \PYG{k+kc}{true}\PYG{p}{,}
            \PYG{n+nt}{\PYGZdq{}deployment\PYGZus{}option\PYGZdq{}}\PYG{p}{:} \PYG{l+s+s2}{\PYGZdq{}small\PYGZdq{}}\PYG{p}{,}
            \PYG{n+nt}{\PYGZdq{}name\PYGZdq{}}\PYG{p}{:} \PYG{l+s+s2}{\PYGZdq{}vcsatest\PYGZdq{}}
        \PYG{p}{\PYGZcb{}}\PYG{p}{,}
        \PYG{n+nt}{\PYGZdq{}network\PYGZdq{}}\PYG{p}{:} \PYG{p}{\PYGZob{}}
            \PYG{n+nt}{\PYGZdq{}ip\PYGZus{}family\PYGZdq{}}\PYG{p}{:} \PYG{l+s+s2}{\PYGZdq{}ipv4\PYGZdq{}}\PYG{p}{,}
            \PYG{n+nt}{\PYGZdq{}mode\PYGZdq{}}\PYG{p}{:} \PYG{l+s+s2}{\PYGZdq{}static\PYGZdq{}}\PYG{p}{,}
            \PYG{n+nt}{\PYGZdq{}ip\PYGZdq{}}\PYG{p}{:} \PYG{l+s+s2}{\PYGZdq{}192.168.101.51\PYGZdq{}}\PYG{p}{,}
            \PYG{n+nt}{\PYGZdq{}dns\PYGZus{}servers\PYGZdq{}}\PYG{p}{:} \PYG{p}{[}
                \PYG{l+s+s2}{\PYGZdq{}114.114.114.114\PYGZdq{}}
            \PYG{p}{]}\PYG{p}{,}
            \PYG{n+nt}{\PYGZdq{}prefix\PYGZdq{}}\PYG{p}{:} \PYG{l+s+s2}{\PYGZdq{}24\PYGZdq{}}\PYG{p}{,}
            \PYG{n+nt}{\PYGZdq{}gateway\PYGZdq{}}\PYG{p}{:} \PYG{l+s+s2}{\PYGZdq{}192.168.101.254\PYGZdq{}}\PYG{p}{,}
            \PYG{n+nt}{\PYGZdq{}system\PYGZus{}name\PYGZdq{}}\PYG{p}{:} \PYG{l+s+s2}{\PYGZdq{}192.168.101.51\PYGZdq{}}
        \PYG{p}{\PYGZcb{}}\PYG{p}{,}
        \PYG{n+nt}{\PYGZdq{}os\PYGZdq{}}\PYG{p}{:} \PYG{p}{\PYGZob{}}
            \PYG{n+nt}{\PYGZdq{}password\PYGZdq{}}\PYG{p}{:} \PYG{l+s+s2}{\PYGZdq{}password\PYGZdq{}}\PYG{p}{,}
            \PYG{n+nt}{\PYGZdq{}ntp\PYGZus{}servers\PYGZdq{}}\PYG{p}{:} \PYG{l+s+s2}{\PYGZdq{}ntp1.aliyun.com\PYGZdq{}}\PYG{p}{,}
            \PYG{n+nt}{\PYGZdq{}ssh\PYGZus{}enable\PYGZdq{}}\PYG{p}{:} \PYG{k+kc}{false}
        \PYG{p}{\PYGZcb{}}\PYG{p}{,}
        \PYG{n+nt}{\PYGZdq{}sso\PYGZdq{}}\PYG{p}{:} \PYG{p}{\PYGZob{}}
            \PYG{n+nt}{\PYGZdq{}password\PYGZdq{}}\PYG{p}{:} \PYG{l+s+s2}{\PYGZdq{}password\PYGZdq{}}\PYG{p}{,}
            \PYG{n+nt}{\PYGZdq{}domain\PYGZus{}name\PYGZdq{}}\PYG{p}{:} \PYG{l+s+s2}{\PYGZdq{}vsphere.local\PYGZdq{}}
        \PYG{p}{\PYGZcb{}}
    \PYG{p}{\PYGZcb{}}\PYG{p}{,}
    \PYG{n+nt}{\PYGZdq{}ceip\PYGZdq{}}\PYG{p}{:} \PYG{p}{\PYGZob{}}
        \PYG{n+nt}{\PYGZdq{}description\PYGZdq{}}\PYG{p}{:} \PYG{p}{\PYGZob{}}
            \PYG{n+nt}{\PYGZdq{}\PYGZus{}\PYGZus{}comments\PYGZdq{}}\PYG{p}{:} \PYG{p}{[}
                \PYG{l+s+s2}{\PYGZdq{}++++VMware Customer Experience Improvement Program (CEIP)++++\PYGZdq{}}\PYG{p}{,}
                \PYG{l+s+s2}{\PYGZdq{}VMware\PYGZsq{}s Customer Experience Improvement Program (CEIP) \PYGZdq{}}\PYG{p}{,}
                \PYG{l+s+s2}{\PYGZdq{}provides VMware with information that enables VMware to \PYGZdq{}}\PYG{p}{,}
                \PYG{l+s+s2}{\PYGZdq{}improve its products and services, to fix problems, \PYGZdq{}}\PYG{p}{,}
                \PYG{l+s+s2}{\PYGZdq{}and to advise you on how best to deploy and use our \PYGZdq{}}\PYG{p}{,}
                \PYG{l+s+s2}{\PYGZdq{}products. As part of CEIP, VMware collects technical \PYGZdq{}}\PYG{p}{,}
                \PYG{l+s+s2}{\PYGZdq{}information about your organization\PYGZsq{}s use of VMware \PYGZdq{}}\PYG{p}{,}
                \PYG{l+s+s2}{\PYGZdq{}products and services on a regular basis in association \PYGZdq{}}\PYG{p}{,}
                \PYG{l+s+s2}{\PYGZdq{}with your organization\PYGZsq{}s VMware license key(s). This \PYGZdq{}}\PYG{p}{,}
                \PYG{l+s+s2}{\PYGZdq{}information does not personally identify any individual. \PYGZdq{}}\PYG{p}{,}
                \PYG{l+s+s2}{\PYGZdq{}\PYGZdq{}}\PYG{p}{,}
                \PYG{l+s+s2}{\PYGZdq{}Additional information regarding the data collected \PYGZdq{}}\PYG{p}{,}
                \PYG{l+s+s2}{\PYGZdq{}through CEIP and the purposes for which it is used by \PYGZdq{}}\PYG{p}{,}
                \PYG{l+s+s2}{\PYGZdq{}VMware is set forth in the Trust \PYGZam{} Assurance Center at \PYGZdq{}}\PYG{p}{,}
                \PYG{l+s+s2}{\PYGZdq{}http://www.vmware.com/trustvmware/ceip.html . If you \PYGZdq{}}\PYG{p}{,}
                \PYG{l+s+s2}{\PYGZdq{}prefer not to participate in VMware\PYGZsq{}s CEIP for this \PYGZdq{}}\PYG{p}{,}
                \PYG{l+s+s2}{\PYGZdq{}product, you should disable CEIP by setting \PYGZdq{}}\PYG{p}{,}
                \PYG{l+s+s2}{\PYGZdq{}\PYGZsq{}ceip\PYGZus{}enabled\PYGZsq{}: false. You may join or leave VMware\PYGZsq{}s \PYGZdq{}}\PYG{p}{,}
                \PYG{l+s+s2}{\PYGZdq{}CEIP for this product at any time. Please confirm your \PYGZdq{}}\PYG{p}{,}
                \PYG{l+s+s2}{\PYGZdq{}acknowledgement by passing in the parameter \PYGZdq{}}\PYG{p}{,}
                \PYG{l+s+s2}{\PYGZdq{}\PYGZhy{}\PYGZhy{}acknowledge\PYGZhy{}ceip in the command line.\PYGZdq{}}\PYG{p}{,}
                \PYG{l+s+s2}{\PYGZdq{}++++++++++++++++++++++++++++++++++++++++++++++++++++++++++++++\PYGZdq{}}
            \PYG{p}{]}
        \PYG{p}{\PYGZcb{}}\PYG{p}{,}
        \PYG{n+nt}{\PYGZdq{}settings\PYGZdq{}}\PYG{p}{:} \PYG{p}{\PYGZob{}}
            \PYG{n+nt}{\PYGZdq{}ceip\PYGZus{}enabled\PYGZdq{}}\PYG{p}{:} \PYG{k+kc}{false}
        \PYG{p}{\PYGZcb{}}
    \PYG{p}{\PYGZcb{}}
\PYG{p}{\PYGZcb{}}
\end{sphinxVerbatim}

\item {} 
\sphinxstylestrong{embedded\_vCSA\_on\_VC.json}

\begin{sphinxVerbatim}[commandchars=\\\{\},numbers=left,firstnumber=1,stepnumber=1]
\PYG{p}{\PYGZob{}}
    \PYG{n+nt}{\PYGZdq{}\PYGZus{}\PYGZus{}version\PYGZdq{}}\PYG{p}{:} \PYG{l+s+s2}{\PYGZdq{}2.13.0\PYGZdq{}}\PYG{p}{,}
    \PYG{n+nt}{\PYGZdq{}\PYGZus{}\PYGZus{}comments\PYGZdq{}}\PYG{p}{:} \PYG{l+s+s2}{\PYGZdq{}Sample template to deploy a vCenter Server Appliance with an embedded Platform Services Controller on a vCenter Server instance.\PYGZdq{}}\PYG{p}{,}
    \PYG{n+nt}{\PYGZdq{}new\PYGZus{}vcsa\PYGZdq{}}\PYG{p}{:} \PYG{p}{\PYGZob{}}
        \PYG{n+nt}{\PYGZdq{}vc\PYGZdq{}}\PYG{p}{:} \PYG{p}{\PYGZob{}}
            \PYG{n+nt}{\PYGZdq{}\PYGZus{}\PYGZus{}comments\PYGZdq{}}\PYG{p}{:} \PYG{p}{[}
                \PYG{l+s+s2}{\PYGZdq{}\PYGZsq{}datacenter\PYGZsq{} must end with a datacenter name, and only with a datacenter name. \PYGZdq{}}\PYG{p}{,}
                \PYG{l+s+s2}{\PYGZdq{}\PYGZsq{}target\PYGZsq{} must end with an ESXi hostname, a cluster name, or a resource pool name. \PYGZdq{}}\PYG{p}{,}
                \PYG{l+s+s2}{\PYGZdq{}The item \PYGZsq{}Resources\PYGZsq{} must precede the resource pool name. \PYGZdq{}}\PYG{p}{,}
                \PYG{l+s+s2}{\PYGZdq{}All names are case\PYGZhy{}sensitive. \PYGZdq{}}\PYG{p}{,}
                \PYG{l+s+s2}{\PYGZdq{}For details and examples, refer to template help, i.e. vcsa\PYGZhy{}deploy \PYGZob{}install\textbar{}upgrade\textbar{}migrate\PYGZcb{} \PYGZhy{}\PYGZhy{}template\PYGZhy{}help\PYGZdq{}}
            \PYG{p}{]}\PYG{p}{,}
            \PYG{n+nt}{\PYGZdq{}hostname\PYGZdq{}}\PYG{p}{:} \PYG{l+s+s2}{\PYGZdq{}192.168.101.100\PYGZdq{}}\PYG{p}{,}
            \PYG{n+nt}{\PYGZdq{}username\PYGZdq{}}\PYG{p}{:} \PYG{l+s+s2}{\PYGZdq{}administrator\PYGZdq{}}\PYG{p}{,}
            \PYG{n+nt}{\PYGZdq{}password\PYGZdq{}}\PYG{p}{:} \PYG{l+s+s2}{\PYGZdq{}1234\PYGZhy{}abcd\PYGZdq{}}\PYG{p}{,}
            \PYG{n+nt}{\PYGZdq{}deployment\PYGZus{}network\PYGZdq{}}\PYG{p}{:} \PYG{l+s+s2}{\PYGZdq{}DP101\PYGZdq{}}\PYG{p}{,}
            \PYG{n+nt}{\PYGZdq{}datacenter\PYGZdq{}}\PYG{p}{:} \PYG{p}{[}
                \PYG{l+s+s2}{\PYGZdq{}testpool01\PYGZdq{}}
            \PYG{p}{]}\PYG{p}{,}
            \PYG{n+nt}{\PYGZdq{}datastore\PYGZdq{}}\PYG{p}{:} \PYG{l+s+s2}{\PYGZdq{}vsanDatastore\PYGZdq{}}\PYG{p}{,}
            \PYG{n+nt}{\PYGZdq{}target\PYGZdq{}}\PYG{p}{:} \PYG{p}{[}
                \PYG{l+s+s2}{\PYGZdq{}cluster01\PYGZdq{}}
            \PYG{p}{]}
        \PYG{p}{\PYGZcb{}}\PYG{p}{,}
     \PYG{n+nt}{\PYGZdq{}appliance\PYGZdq{}}\PYG{p}{:} \PYG{p}{\PYGZob{}}
            \PYG{n+nt}{\PYGZdq{}\PYGZus{}\PYGZus{}comments\PYGZdq{}}\PYG{p}{:} \PYG{p}{[}
                \PYG{l+s+s2}{\PYGZdq{}You must provide the \PYGZsq{}deployment\PYGZus{}option\PYGZsq{} key with a value, which will affect the VCSA\PYGZsq{}s configuration parameters, such as the VCSA\PYGZsq{}s number of vCPUs, the memory size, the storage size, and the maximum numbers of ESXi hosts and VMs which can be managed. For a list of acceptable values, run the supported deployment sizes help, i.e. vcsa\PYGZhy{}deploy \PYGZhy{}\PYGZhy{}supported\PYGZhy{}deployment\PYGZhy{}sizes\PYGZdq{}}
            \PYG{p}{]}\PYG{p}{,}
            \PYG{n+nt}{\PYGZdq{}thin\PYGZus{}disk\PYGZus{}mode\PYGZdq{}}\PYG{p}{:} \PYG{k+kc}{true}\PYG{p}{,}
            \PYG{n+nt}{\PYGZdq{}deployment\PYGZus{}option\PYGZdq{}}\PYG{p}{:} \PYG{l+s+s2}{\PYGZdq{}small\PYGZdq{}}\PYG{p}{,}
            \PYG{n+nt}{\PYGZdq{}name\PYGZdq{}}\PYG{p}{:} \PYG{l+s+s2}{\PYGZdq{}vcsatest\PYGZdq{}}
        \PYG{p}{\PYGZcb{}}\PYG{p}{,}
        \PYG{n+nt}{\PYGZdq{}network\PYGZdq{}}\PYG{p}{:} \PYG{p}{\PYGZob{}}
            \PYG{n+nt}{\PYGZdq{}ip\PYGZus{}family\PYGZdq{}}\PYG{p}{:} \PYG{l+s+s2}{\PYGZdq{}ipv4\PYGZdq{}}\PYG{p}{,}
            \PYG{n+nt}{\PYGZdq{}mode\PYGZdq{}}\PYG{p}{:} \PYG{l+s+s2}{\PYGZdq{}static\PYGZdq{}}\PYG{p}{,}
            \PYG{n+nt}{\PYGZdq{}ip\PYGZdq{}}\PYG{p}{:} \PYG{l+s+s2}{\PYGZdq{}192.168.101.51\PYGZdq{}}\PYG{p}{,}
            \PYG{n+nt}{\PYGZdq{}dns\PYGZus{}servers\PYGZdq{}}\PYG{p}{:} \PYG{p}{[}
                \PYG{l+s+s2}{\PYGZdq{}114.114.114.114\PYGZdq{}}
            \PYG{p}{]}\PYG{p}{,}
            \PYG{n+nt}{\PYGZdq{}prefix\PYGZdq{}}\PYG{p}{:} \PYG{l+s+s2}{\PYGZdq{}24\PYGZdq{}}\PYG{p}{,}
            \PYG{n+nt}{\PYGZdq{}gateway\PYGZdq{}}\PYG{p}{:} \PYG{l+s+s2}{\PYGZdq{}192.168.101.254\PYGZdq{}}\PYG{p}{,}
            \PYG{n+nt}{\PYGZdq{}system\PYGZus{}name\PYGZdq{}}\PYG{p}{:} \PYG{l+s+s2}{\PYGZdq{}192.168.101.51\PYGZdq{}}
        \PYG{p}{\PYGZcb{}}\PYG{p}{,}
        \PYG{n+nt}{\PYGZdq{}os\PYGZdq{}}\PYG{p}{:} \PYG{p}{\PYGZob{}}
            \PYG{n+nt}{\PYGZdq{}password\PYGZdq{}}\PYG{p}{:} \PYG{l+s+s2}{\PYGZdq{}tR@aC8\PYGZsh{}3!\PYGZdq{}}\PYG{p}{,}
            \PYG{n+nt}{\PYGZdq{}ntp\PYGZus{}servers\PYGZdq{}}\PYG{p}{:} \PYG{l+s+s2}{\PYGZdq{}ntp1.aliyun.com\PYGZdq{}}\PYG{p}{,}
            \PYG{n+nt}{\PYGZdq{}ssh\PYGZus{}enable\PYGZdq{}}\PYG{p}{:} \PYG{k+kc}{false}
        \PYG{p}{\PYGZcb{}}\PYG{p}{,}
        \PYG{n+nt}{\PYGZdq{}sso\PYGZdq{}}\PYG{p}{:} \PYG{p}{\PYGZob{}}
            \PYG{n+nt}{\PYGZdq{}password\PYGZdq{}}\PYG{p}{:} \PYG{l+s+s2}{\PYGZdq{}tR@aC8\PYGZsh{}3!\PYGZdq{}}\PYG{p}{,}
            \PYG{n+nt}{\PYGZdq{}domain\PYGZus{}name\PYGZdq{}}\PYG{p}{:} \PYG{l+s+s2}{\PYGZdq{}vsphere.local\PYGZdq{}}
        \PYG{p}{\PYGZcb{}}
    \PYG{p}{\PYGZcb{}}\PYG{p}{,}
    \PYG{n+nt}{\PYGZdq{}ceip\PYGZdq{}}\PYG{p}{:} \PYG{p}{\PYGZob{}}
        \PYG{n+nt}{\PYGZdq{}description\PYGZdq{}}\PYG{p}{:} \PYG{p}{\PYGZob{}}
            \PYG{n+nt}{\PYGZdq{}\PYGZus{}\PYGZus{}comments\PYGZdq{}}\PYG{p}{:} \PYG{p}{[}
                \PYG{l+s+s2}{\PYGZdq{}++++VMware Customer Experience Improvement Program (CEIP)++++\PYGZdq{}}\PYG{p}{,}
                \PYG{l+s+s2}{\PYGZdq{}VMware\PYGZsq{}s Customer Experience Improvement Program (CEIP) \PYGZdq{}}\PYG{p}{,}
                \PYG{l+s+s2}{\PYGZdq{}provides VMware with information that enables VMware to \PYGZdq{}}\PYG{p}{,}
                \PYG{l+s+s2}{\PYGZdq{}improve its products and services, to fix problems, \PYGZdq{}}\PYG{p}{,}
                \PYG{l+s+s2}{\PYGZdq{}and to advise you on how best to deploy and use our \PYGZdq{}}\PYG{p}{,}
                \PYG{l+s+s2}{\PYGZdq{}products. As part of CEIP, VMware collects technical \PYGZdq{}}\PYG{p}{,}
                \PYG{l+s+s2}{\PYGZdq{}information about your organization\PYGZsq{}s use of VMware \PYGZdq{}}\PYG{p}{,}
                \PYG{l+s+s2}{\PYGZdq{}products and services on a regular basis in association \PYGZdq{}}\PYG{p}{,}
                \PYG{l+s+s2}{\PYGZdq{}with your organization\PYGZsq{}s VMware license key(s). This \PYGZdq{}}\PYG{p}{,}
                \PYG{l+s+s2}{\PYGZdq{}information does not personally identify any individual. \PYGZdq{}}\PYG{p}{,}
                \PYG{l+s+s2}{\PYGZdq{}\PYGZdq{}}\PYG{p}{,}
                \PYG{l+s+s2}{\PYGZdq{}Additional information regarding the data collected \PYGZdq{}}\PYG{p}{,}
                \PYG{l+s+s2}{\PYGZdq{}through CEIP and the purposes for which it is used by \PYGZdq{}}\PYG{p}{,}
                \PYG{l+s+s2}{\PYGZdq{}VMware is set forth in the Trust \PYGZam{} Assurance Center at \PYGZdq{}}\PYG{p}{,}
                \PYG{l+s+s2}{\PYGZdq{}http://www.vmware.com/trustvmware/ceip.html . If you \PYGZdq{}}\PYG{p}{,}
                \PYG{l+s+s2}{\PYGZdq{}prefer not to participate in VMware\PYGZsq{}s CEIP for this \PYGZdq{}}\PYG{p}{,}
                \PYG{l+s+s2}{\PYGZdq{}product, you should disable CEIP by setting \PYGZdq{}}\PYG{p}{,}
                \PYG{l+s+s2}{\PYGZdq{}\PYGZsq{}ceip\PYGZus{}enabled\PYGZsq{}: false. You may join or leave VMware\PYGZsq{}s \PYGZdq{}}\PYG{p}{,}
                \PYG{l+s+s2}{\PYGZdq{}CEIP for this product at any time. Please confirm your \PYGZdq{}}\PYG{p}{,}
                \PYG{l+s+s2}{\PYGZdq{}acknowledgement by passing in the parameter \PYGZdq{}}\PYG{p}{,}
                \PYG{l+s+s2}{\PYGZdq{}\PYGZhy{}\PYGZhy{}acknowledge\PYGZhy{}ceip in the command line.\PYGZdq{}}\PYG{p}{,}
                \PYG{l+s+s2}{\PYGZdq{}++++++++++++++++++++++++++++++++++++++++++++++++++++++++++++++\PYGZdq{}}
            \PYG{p}{]}
        \PYG{p}{\PYGZcb{}}\PYG{p}{,}
        \PYG{n+nt}{\PYGZdq{}settings\PYGZdq{}}\PYG{p}{:} \PYG{p}{\PYGZob{}}
            \PYG{n+nt}{\PYGZdq{}ceip\PYGZus{}enabled\PYGZdq{}}\PYG{p}{:} \PYG{k+kc}{true}
        \PYG{p}{\PYGZcb{}}
    \PYG{p}{\PYGZcb{}}
\PYG{p}{\PYGZcb{}}
\end{sphinxVerbatim}

\item {} 
\sphinxstylestrong{embedded\_vCSA\_replication\_on\_VC.json}

\begin{sphinxVerbatim}[commandchars=\\\{\},numbers=left,firstnumber=1,stepnumber=1]
\PYG{p}{\PYGZob{}}
    \PYG{n+nt}{\PYGZdq{}\PYGZus{}\PYGZus{}version\PYGZdq{}}\PYG{p}{:} \PYG{l+s+s2}{\PYGZdq{}2.13.0\PYGZdq{}}\PYG{p}{,}
    \PYG{n+nt}{\PYGZdq{}\PYGZus{}\PYGZus{}comments\PYGZdq{}}\PYG{p}{:} \PYG{l+s+s2}{\PYGZdq{}Sample template to deploy a vCenter Server Appliance with an embedded Platform Services Controller on a vCenter Server instance.\PYGZdq{}}\PYG{p}{,}
    \PYG{n+nt}{\PYGZdq{}new\PYGZus{}vcsa\PYGZdq{}}\PYG{p}{:} \PYG{p}{\PYGZob{}}
        \PYG{n+nt}{\PYGZdq{}vc\PYGZdq{}}\PYG{p}{:} \PYG{p}{\PYGZob{}}
            \PYG{n+nt}{\PYGZdq{}\PYGZus{}\PYGZus{}comments\PYGZdq{}}\PYG{p}{:} \PYG{p}{[}
                \PYG{l+s+s2}{\PYGZdq{}\PYGZsq{}datacenter\PYGZsq{} must end with a datacenter name, and only with a datacenter name. \PYGZdq{}}\PYG{p}{,}
                \PYG{l+s+s2}{\PYGZdq{}\PYGZsq{}target\PYGZsq{} must end with an ESXi hostname, a cluster name, or a resource pool name. \PYGZdq{}}\PYG{p}{,}
                \PYG{l+s+s2}{\PYGZdq{}The item \PYGZsq{}Resources\PYGZsq{} must precede the resource pool name. \PYGZdq{}}\PYG{p}{,}
                \PYG{l+s+s2}{\PYGZdq{}All names are case\PYGZhy{}sensitive. \PYGZdq{}}\PYG{p}{,}
                \PYG{l+s+s2}{\PYGZdq{}For details and examples, refer to template help, i.e. vcsa\PYGZhy{}deploy \PYGZob{}install\textbar{}upgrade\textbar{}migrate\PYGZcb{} \PYGZhy{}\PYGZhy{}template\PYGZhy{}help\PYGZdq{}}
            \PYG{p}{]}\PYG{p}{,}
            \PYG{n+nt}{\PYGZdq{}hostname\PYGZdq{}}\PYG{p}{:} \PYG{l+s+s2}{\PYGZdq{}192.168.101.100\PYGZdq{}}\PYG{p}{,}
            \PYG{n+nt}{\PYGZdq{}username\PYGZdq{}}\PYG{p}{:} \PYG{l+s+s2}{\PYGZdq{}administrator\PYGZdq{}}\PYG{p}{,}
            \PYG{n+nt}{\PYGZdq{}password\PYGZdq{}}\PYG{p}{:} \PYG{l+s+s2}{\PYGZdq{}password\PYGZdq{}}\PYG{p}{,}
            \PYG{n+nt}{\PYGZdq{}deployment\PYGZus{}network\PYGZdq{}}\PYG{p}{:} \PYG{l+s+s2}{\PYGZdq{}DP101\PYGZdq{}}\PYG{p}{,}
            \PYG{n+nt}{\PYGZdq{}datacenter\PYGZdq{}}\PYG{p}{:} \PYG{p}{[}
                \PYG{l+s+s2}{\PYGZdq{}testpool01\PYGZdq{}}
            \PYG{p}{]}\PYG{p}{,}
            \PYG{n+nt}{\PYGZdq{}datastore\PYGZdq{}}\PYG{p}{:} \PYG{l+s+s2}{\PYGZdq{}vsanDatastore\PYGZdq{}}\PYG{p}{,}
            \PYG{n+nt}{\PYGZdq{}target\PYGZdq{}}\PYG{p}{:} \PYG{p}{[}
                \PYG{l+s+s2}{\PYGZdq{}cluster01\PYGZdq{}}
            \PYG{p}{]}
        \PYG{p}{\PYGZcb{}}\PYG{p}{,}
     \PYG{n+nt}{\PYGZdq{}appliance\PYGZdq{}}\PYG{p}{:} \PYG{p}{\PYGZob{}}
            \PYG{n+nt}{\PYGZdq{}\PYGZus{}\PYGZus{}comments\PYGZdq{}}\PYG{p}{:} \PYG{p}{[}
                \PYG{l+s+s2}{\PYGZdq{}You must provide the \PYGZsq{}deployment\PYGZus{}option\PYGZsq{} key with a value, which will affect the VCSA\PYGZsq{}s configuration parameters, such as the VCSA\PYGZsq{}s number of vCPUs, the memory size, the storage size, and the maximum numbers of ESXi hosts and VMs which can be managed. For a list of acceptable values, run the supported deployment sizes help, i.e. vcsa\PYGZhy{}deploy \PYGZhy{}\PYGZhy{}supported\PYGZhy{}deployment\PYGZhy{}sizes\PYGZdq{}}
            \PYG{p}{]}\PYG{p}{,}
            \PYG{n+nt}{\PYGZdq{}thin\PYGZus{}disk\PYGZus{}mode\PYGZdq{}}\PYG{p}{:} \PYG{k+kc}{true}\PYG{p}{,}
            \PYG{n+nt}{\PYGZdq{}deployment\PYGZus{}option\PYGZdq{}}\PYG{p}{:} \PYG{l+s+s2}{\PYGZdq{}small\PYGZdq{}}\PYG{p}{,}
            \PYG{n+nt}{\PYGZdq{}name\PYGZdq{}}\PYG{p}{:} \PYG{l+s+s2}{\PYGZdq{}vCenter\PYGZhy{}Server\PYGZhy{}Appliance\PYGZus{}rep\PYGZdq{}}
        \PYG{p}{\PYGZcb{}}\PYG{p}{,}
        \PYG{n+nt}{\PYGZdq{}network\PYGZdq{}}\PYG{p}{:} \PYG{p}{\PYGZob{}}
            \PYG{n+nt}{\PYGZdq{}ip\PYGZus{}family\PYGZdq{}}\PYG{p}{:} \PYG{l+s+s2}{\PYGZdq{}ipv4\PYGZdq{}}\PYG{p}{,}
            \PYG{n+nt}{\PYGZdq{}mode\PYGZdq{}}\PYG{p}{:} \PYG{l+s+s2}{\PYGZdq{}static\PYGZdq{}}\PYG{p}{,}
            \PYG{n+nt}{\PYGZdq{}ip\PYGZdq{}}\PYG{p}{:} \PYG{l+s+s2}{\PYGZdq{}192.168.101.101\PYGZdq{}}\PYG{p}{,}
            \PYG{n+nt}{\PYGZdq{}dns\PYGZus{}servers\PYGZdq{}}\PYG{p}{:} \PYG{p}{[}
                \PYG{l+s+s2}{\PYGZdq{}114.114.114.114\PYGZdq{}}
            \PYG{p}{]}\PYG{p}{,}
            \PYG{n+nt}{\PYGZdq{}prefix\PYGZdq{}}\PYG{p}{:} \PYG{l+s+s2}{\PYGZdq{}24\PYGZdq{}}\PYG{p}{,}
            \PYG{n+nt}{\PYGZdq{}gateway\PYGZdq{}}\PYG{p}{:} \PYG{l+s+s2}{\PYGZdq{}192.168.101.254\PYGZdq{}}\PYG{p}{,}
            \PYG{n+nt}{\PYGZdq{}system\PYGZus{}name\PYGZdq{}}\PYG{p}{:} \PYG{l+s+s2}{\PYGZdq{}192.168.101.101\PYGZdq{}}
        \PYG{p}{\PYGZcb{}}\PYG{p}{,}
        \PYG{n+nt}{\PYGZdq{}os\PYGZdq{}}\PYG{p}{:} \PYG{p}{\PYGZob{}}
            \PYG{n+nt}{\PYGZdq{}password\PYGZdq{}}\PYG{p}{:} \PYG{l+s+s2}{\PYGZdq{}password\PYGZdq{}}\PYG{p}{,}
            \PYG{n+nt}{\PYGZdq{}ntp\PYGZus{}servers\PYGZdq{}}\PYG{p}{:} \PYG{l+s+s2}{\PYGZdq{}ntp1.aliyun.com\PYGZdq{}}\PYG{p}{,}
            \PYG{n+nt}{\PYGZdq{}ssh\PYGZus{}enable\PYGZdq{}}\PYG{p}{:} \PYG{k+kc}{false}
        \PYG{p}{\PYGZcb{}}\PYG{p}{,}
        \PYG{n+nt}{\PYGZdq{}sso\PYGZdq{}}\PYG{p}{:} \PYG{p}{\PYGZob{}}
            \PYG{n+nt}{\PYGZdq{}password\PYGZdq{}}\PYG{p}{:} \PYG{l+s+s2}{\PYGZdq{}password\PYGZdq{}}\PYG{p}{,}
            \PYG{n+nt}{\PYGZdq{}domain\PYGZus{}name\PYGZdq{}}\PYG{p}{:} \PYG{l+s+s2}{\PYGZdq{}vsphere.local\PYGZdq{}}\PYG{p}{,}
            \PYG{n+nt}{\PYGZdq{}first\PYGZus{}instance\PYGZdq{}}\PYG{p}{:} \PYG{k+kc}{false}\PYG{p}{,}
            \PYG{n+nt}{\PYGZdq{}replication\PYGZus{}partner\PYGZus{}hostname\PYGZdq{}}\PYG{p}{:} \PYG{l+s+s2}{\PYGZdq{}192.168.101.100\PYGZdq{}}\PYG{p}{,}
            \PYG{n+nt}{\PYGZdq{}sso\PYGZus{}port\PYGZdq{}}\PYG{p}{:} \PYG{l+m+mi}{443}
        \PYG{p}{\PYGZcb{}}
    \PYG{p}{\PYGZcb{}}\PYG{p}{,}
    \PYG{n+nt}{\PYGZdq{}ceip\PYGZdq{}}\PYG{p}{:} \PYG{p}{\PYGZob{}}
        \PYG{n+nt}{\PYGZdq{}description\PYGZdq{}}\PYG{p}{:} \PYG{p}{\PYGZob{}}
            \PYG{n+nt}{\PYGZdq{}\PYGZus{}\PYGZus{}comments\PYGZdq{}}\PYG{p}{:} \PYG{p}{[}
                \PYG{l+s+s2}{\PYGZdq{}++++VMware Customer Experience Improvement Program (CEIP)++++\PYGZdq{}}\PYG{p}{,}
                \PYG{l+s+s2}{\PYGZdq{}VMware\PYGZsq{}s Customer Experience Improvement Program (CEIP) \PYGZdq{}}\PYG{p}{,}
                \PYG{l+s+s2}{\PYGZdq{}provides VMware with information that enables VMware to \PYGZdq{}}\PYG{p}{,}
                \PYG{l+s+s2}{\PYGZdq{}improve its products and services, to fix problems, \PYGZdq{}}\PYG{p}{,}
                \PYG{l+s+s2}{\PYGZdq{}and to advise you on how best to deploy and use our \PYGZdq{}}\PYG{p}{,}
                \PYG{l+s+s2}{\PYGZdq{}products. As part of CEIP, VMware collects technical \PYGZdq{}}\PYG{p}{,}
                \PYG{l+s+s2}{\PYGZdq{}information about your organization\PYGZsq{}s use of VMware \PYGZdq{}}\PYG{p}{,}
                \PYG{l+s+s2}{\PYGZdq{}products and services on a regular basis in association \PYGZdq{}}\PYG{p}{,}
                \PYG{l+s+s2}{\PYGZdq{}with your organization\PYGZsq{}s VMware license key(s). This \PYGZdq{}}\PYG{p}{,}
                \PYG{l+s+s2}{\PYGZdq{}information does not personally identify any individual. \PYGZdq{}}\PYG{p}{,}
                \PYG{l+s+s2}{\PYGZdq{}\PYGZdq{}}\PYG{p}{,}
                \PYG{l+s+s2}{\PYGZdq{}Additional information regarding the data collected \PYGZdq{}}\PYG{p}{,}
                \PYG{l+s+s2}{\PYGZdq{}through CEIP and the purposes for which it is used by \PYGZdq{}}\PYG{p}{,}
                \PYG{l+s+s2}{\PYGZdq{}VMware is set forth in the Trust \PYGZam{} Assurance Center at \PYGZdq{}}\PYG{p}{,}
                \PYG{l+s+s2}{\PYGZdq{}http://www.vmware.com/trustvmware/ceip.html . If you \PYGZdq{}}\PYG{p}{,}
                \PYG{l+s+s2}{\PYGZdq{}prefer not to participate in VMware\PYGZsq{}s CEIP for this \PYGZdq{}}\PYG{p}{,}
                \PYG{l+s+s2}{\PYGZdq{}product, you should disable CEIP by setting \PYGZdq{}}\PYG{p}{,}
                \PYG{l+s+s2}{\PYGZdq{}\PYGZsq{}ceip\PYGZus{}enabled\PYGZsq{}: false. You may join or leave VMware\PYGZsq{}s \PYGZdq{}}\PYG{p}{,}
                \PYG{l+s+s2}{\PYGZdq{}CEIP for this product at any time. Please confirm your \PYGZdq{}}\PYG{p}{,}
                \PYG{l+s+s2}{\PYGZdq{}acknowledgement by passing in the parameter \PYGZdq{}}\PYG{p}{,}
                \PYG{l+s+s2}{\PYGZdq{}\PYGZhy{}\PYGZhy{}acknowledge\PYGZhy{}ceip in the command line.\PYGZdq{}}\PYG{p}{,}
                \PYG{l+s+s2}{\PYGZdq{}++++++++++++++++++++++++++++++++++++++++++++++++++++++++++++++\PYGZdq{}}
            \PYG{p}{]}
        \PYG{p}{\PYGZcb{}}\PYG{p}{,}
        \PYG{n+nt}{\PYGZdq{}settings\PYGZdq{}}\PYG{p}{:} \PYG{p}{\PYGZob{}}
            \PYG{n+nt}{\PYGZdq{}ceip\PYGZus{}enabled\PYGZdq{}}\PYG{p}{:} \PYG{k+kc}{false}
        \PYG{p}{\PYGZcb{}}
    \PYG{p}{\PYGZcb{}}
\PYG{p}{\PYGZcb{}}
\end{sphinxVerbatim}

\end{itemize}


\chapter{Python}
\label{\detokenize{python/index:python}}\label{\detokenize{python/index::doc}}

\section{pyenv安装}
\label{\detokenize{python/pyenv:pyenv}}\label{\detokenize{python/pyenv::doc}}\begin{itemize}
\item {} 
pyenv 安装

\end{itemize}
\begin{quote}

\begin{sphinxVerbatim}[commandchars=\\\{\}]
yum install zlib\PYGZhy{}devel bzip2\PYGZhy{}devel openssl\PYGZhy{}devel ncurses\PYGZhy{}devel sqlite\PYGZhy{}devel readline\PYGZhy{}devel tk\PYGZhy{}devel gdbm\PYGZhy{}devel db4\PYGZhy{}devel libpcap\PYGZhy{}devel xz\PYGZhy{}devel git \PYGZhy{}y
mkdir \PYGZti{}/.pyenv
git clone git://github.com/yyuu/pyenv.git \PYGZti{}/.pyenv
\PYG{n+nb}{echo} \PYG{l+s+s1}{\PYGZsq{}export PYENV\PYGZus{}ROOT=\PYGZdq{}\PYGZdl{}HOME/.pyenv\PYGZdq{}\PYGZsq{}} \PYGZgt{}\PYGZgt{} \PYGZti{}/.bashrc
\PYG{n+nb}{echo} \PYG{l+s+s1}{\PYGZsq{}export PATH=\PYGZdq{}\PYGZdl{}PYENV\PYGZus{}ROOT/bin:\PYGZdl{}PATH\PYGZdq{}\PYGZsq{}} \PYGZgt{}\PYGZgt{} \PYGZti{}/.bashrc
\PYG{n+nb}{echo} \PYG{l+s+s1}{\PYGZsq{}eval \PYGZdq{}\PYGZdl{}(pyenv init \PYGZhy{})\PYGZdq{}\PYGZsq{}} \PYGZgt{}\PYGZgt{} \PYGZti{}/.bashrc
\PYG{n+nb}{exec} \PYG{n+nv}{\PYGZdl{}SHELL} \PYGZhy{}l
\PYG{c+c1}{\PYGZsh{}加速下载}
\PYG{n+nv}{v}\PYG{o}{=}\PYG{l+m}{3}.6.5\PYG{p}{;}wget http://mirrors.sohu.com/python/\PYG{n+nv}{\PYGZdl{}v}/Python\PYGZhy{}\PYG{n+nv}{\PYGZdl{}v}.tar.xz \PYGZhy{}P \PYGZti{}/.pyenv/cache/\PYG{p}{;}pyenv install \PYG{n+nv}{\PYGZdl{}v}
\PYG{c+c1}{\PYGZsh{}gitlab推荐方式}
curl \PYGZhy{}L https://github.com/pyenv/pyenv\PYGZhy{}installer/raw/master/bin/pyenv\PYGZhy{}installer \PYG{p}{\textbar{}} bash
pyenv update
\end{sphinxVerbatim}
\end{quote}
\begin{itemize}
\item {} 
virtualenv安装

\end{itemize}
\begin{quote}

\begin{sphinxVerbatim}[commandchars=\\\{\}]
git clone https://github.com/yyuu/pyenv\PYGZhy{}virtualenv.git  \PYGZti{}/.pyenv/plugins/pyenv\PYGZhy{}virtualenv
\PYG{n+nb}{echo} \PYG{l+s+s1}{\PYGZsq{}eval \PYGZdq{}\PYGZdl{}(pyenv virtualenv\PYGZhy{}init \PYGZhy{})\PYGZdq{}\PYGZsq{}} \PYGZgt{}\PYGZgt{} \PYGZti{}/.bash\PYGZus{}profile
\PYG{n+nb}{source} \PYGZti{}/.bash\PYGZus{}profile
\end{sphinxVerbatim}
\end{quote}


\chapter{网络}
\label{\detokenize{network/index:id1}}\label{\detokenize{network/index::doc}}

\section{华为交换机配置}
\label{\detokenize{network/huawei:id1}}\label{\detokenize{network/huawei::doc}}\begin{itemize}
\item {} 
BootRom重置

\begin{sphinxVerbatim}[commandchars=\\\{\}]
出现\PYGZdq{}Press Ctrl+B to Enter Boot Menu...\PYGZdq{}打印信息时,请在5秒钟内按下\PYGZdq{}Ctrl+B\PYGZdq{},
输入BootROM密码\PYGZdq{}O\PYGZam{}m15213\PYGZdq{}(USG)进入BootROM主系统菜单
交换机V100R005的BootRom密码是huawei
V1R6C05之后版本 BootRom密码Admin@huawei.com
\end{sphinxVerbatim}

\item {} 
SSH配置

\begin{sphinxVerbatim}[commandchars=\\\{\}]
\PYG{n}{stelnet} \PYG{n}{server} \PYG{n}{enable}
\PYG{n}{ssh} \PYG{n}{authentication}\PYG{o}{\PYGZhy{}}\PYG{n+nb}{type} \PYG{n}{default} \PYG{n}{password}
\PYG{n}{user}\PYG{o}{\PYGZhy{}}\PYG{n}{interface} \PYG{n}{vty} \PYG{l+m+mi}{0} \PYG{l+m+mi}{4}
 \PYG{n}{protocol} \PYG{n}{inbound} \PYG{n+nb}{all}
 \PYG{n}{rsa} \PYG{n}{local}\PYG{o}{\PYGZhy{}}\PYG{n}{key}\PYG{o}{\PYGZhy{}}\PYG{n}{pair} \PYG{n}{create}
\PYG{n}{aaa}
 \PYG{n}{local}\PYG{o}{\PYGZhy{}}\PYG{n}{user} \PYG{n}{admin} \PYG{n}{service}\PYG{o}{\PYGZhy{}}\PYG{n+nb}{type} \PYG{n}{telnet} \PYG{n}{http} \PYG{n}{ssh}
\end{sphinxVerbatim}

\item {} 
Tacacs配置

\begin{sphinxVerbatim}[commandchars=\\\{\}]
hwtacscs\PYGZhy{}server template acs
hwtacacs\PYGZhy{}server authentication 10.51.80.108
hwtacacs\PYGZhy{}server authorization 10.51.80.108
hwtacacs\PYGZhy{}server accounting 10.51.80.108
hwtacacs\PYGZhy{}server shared\PYGZhy{}key cipher \PYGZpc{}@\PYGZpc{}@\PYGZdl{}\PYGZcb{}kNTI\PYGZdq{}@!;HLYEAz*r.OP@gt\PYGZpc{}@\PYGZpc{}@
aaa
authentication\PYGZhy{}scheme acs
  authentication\PYGZhy{}mode hwtacacs local
authorization\PYGZhy{}scheme acs
  authorization\PYGZhy{}mode hwtacacs local
  authorization\PYGZhy{}cmd 0 hwtacacs local
  authorization\PYGZhy{}cmd 1 hwtacacs local
  authorization\PYGZhy{}cmd 2 hwtacacs local
  authorization\PYGZhy{}cmd 3 hwtacacs local
accounting\PYGZhy{}scheme acs
  accounting\PYGZhy{}mode hwtacacs
domain default\PYGZus{}admin
  authentication\PYGZhy{}scheme acs
  accounting\PYGZhy{}scheme acs
  authorization\PYGZhy{}scheme acs
  hwtacacs\PYGZhy{}server acs
\PYGZsh{} usg domain 配置
domain acs
  authentication\PYGZhy{}scheme hwtacacs
  accounting\PYGZhy{}scheme acs
  authorization\PYGZhy{}scheme acs
  hwtacacs\PYGZhy{}server acs
  service\PYGZhy{}type internetaccess ssl\PYGZhy{}vpn l2tp ike administrator\PYGZhy{}access
  internet\PYGZhy{}access mode password
  reference user current\PYGZhy{}domain
  new\PYGZhy{}user add\PYGZhy{}temporary group /acs
\end{sphinxVerbatim}

\end{itemize}


\chapter{Indices and tables}
\label{\detokenize{index:indices-and-tables}}\begin{itemize}
\item {} 
\DUrole{xref,std,std-ref}{genindex}

\item {} 
\DUrole{xref,std,std-ref}{modindex}

\end{itemize}



\renewcommand{\indexname}{Index}
\printindex
\end{document}